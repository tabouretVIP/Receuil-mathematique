% ========================================================================
% PACKAGES ESSENTIELS
% ========================================================================

% Encodage et langues
\usepackage[utf8]{inputenc}
\usepackage[T1]{fontenc}
\usepackage[french]{babel}  % Support du français

% Polices élégantes
\usepackage{palatino}            % Police principale élégante (Palatino)
\usepackage{mathpazo}            % Mathématiques assorties à Palatino

% Géométrie et mise en page
\usepackage[
    a4paper,
    inner=2.5cm,
    outer=2cm,
    top=2.5cm,
    bottom=2.5cm,
    headheight=15pt
]{geometry}

% En-têtes et pieds de page élégants
\usepackage{scrlayer-scrpage}
% Tables des matières
\usepackage{etoc}
% Graphiques et couleurs
\usepackage{xcolor}
\usepackage{tikz}
\usepackage{graphicx}
\usetikzlibrary{calc,decorations.pathmorphing,patterns,positioning,shadows}

% Définition des couleurs du thème
\definecolor{primarycolor}{RGB}{0,51,102}      % Bleu foncé
\definecolor{secondarycolor}{RGB}{204,51,51}   % Rouge profond
\definecolor{accentcolor}{RGB}{0,102,204}      % Bleu vif
\definecolor{lightgray}{RGB}{245,245,245}      % Gris très clair
\definecolor{darkgray}{RGB}{80,80,80}          % Gris foncé pour les définitions
\definecolor{lemmecolor}{RGB}{204,102,0}       % Orange foncé pour les lemmes
\definecolor{axiomcolor}{RGB}{102,51,153}      % Violet pour axiomes
\definecolor{examplecolor}{RGB}{34,139,34}     % Vert pour exemples
\definecolor{exercisecolor}{RGB}{255,140,0}    % Orange vif pour exercices

% Mathématiques
\usepackage{amsmath,amsthm,amssymb,amsfonts}
\usepackage{mathtools}
\usepackage{thmtools}
\usepackage{mathrsfs}

% Boîtes colorées pour les théorèmes
\usepackage[most]{tcolorbox}
\tcbuselibrary{theorems,skins,breakable}

% Liens hypertextes (à charger en dernier ou presque)
\usepackage{hyperref}
\hypersetup{
    colorlinks=true,
    linkcolor=primarycolor,
    urlcolor=accentcolor,
    citecolor=secondarycolor,
    pdfborder={0 0 0}
}

% ========================================================================
% ENVIRONNEMENTS DE THÉORÈMES - VERSION AVEC SYNTAXE []
% ========================================================================

% Compteur partagé pour numérotation cohérente
\newcounter{tcbthm}[section]
\renewcommand{\thetcbthm}{\thesection.\arabic{tcbthm}}

% Style de base pour les théorèmes
\tcbset{
    mytheoremstyle/.style={
        enhanced,
        breakable,
        frame hidden,
        colback=white,
        fonttitle=\bfseries\sffamily,
        before skip=12pt,
        after skip=12pt,
        left=8pt,
        right=5pt,
        top=5pt,
        bottom=5pt,
    }
}

% Théorème (barre bleue foncée)
\newtcolorbox{theoreme}[1][]{
    mytheoremstyle,
    borderline west={2pt}{0pt}{primarycolor},
    coltitle=primarycolor,
    title={\refstepcounter{tcbthm}Théorème~\thetcbthm\if\relax\detokenize{#1}\relax\else\ (#1)\fi},
}

% Proposition (barre bleue vive)
\newtcolorbox{proposition}[1][]{
    mytheoremstyle,
    borderline west={2pt}{0pt}{accentcolor},
    coltitle=accentcolor,
    title={\refstepcounter{tcbthm}Proposition~\thetcbthm\if\relax\detokenize{#1}\relax\else\ (#1)\fi},
}

% Lemme (barre orange foncé)
\newtcolorbox{lemme}[1][]{
    mytheoremstyle,
    borderline west={2pt}{0pt}{lemmecolor},
    coltitle=lemmecolor,
    title={\refstepcounter{tcbthm}Lemme~\thetcbthm\if\relax\detokenize{#1}\relax\else\ (#1)\fi},
}

% Corollaire (barre bleue atténuée)
\newtcolorbox{corollaire}[1][]{
    mytheoremstyle,
    borderline west={2pt}{0pt}{primarycolor!70},
    coltitle=primarycolor!70,
    title={\refstepcounter{tcbthm}Corollaire~\thetcbthm\if\relax\detokenize{#1}\relax\else\ (#1)\fi},
}

% Définition (barre grise foncée)
\newtcolorbox{definition}[1][]{
    mytheoremstyle,
    borderline west={2pt}{0pt}{darkgray},
    coltitle=darkgray,
    title={\refstepcounter{tcbthm}Définition~\thetcbthm\if\relax\detokenize{#1}\relax\else\ (#1)\fi},
}

% Axiome (barre violette)
\newtcolorbox{axiome}[1][]{
    mytheoremstyle,
    borderline west={2pt}{0pt}{axiomcolor},
    coltitle=axiomcolor,
    title={\refstepcounter{tcbthm}Axiome~\thetcbthm\if\relax\detokenize{#1}\relax\else\ (#1)\fi},
}

% Conjecture (barre violette)
\newtcolorbox{conjecture}[1][]{
    mytheoremstyle,
    borderline west={2pt}{0pt}{axiomcolor},
    coltitle=axiomcolor,
    title={\refstepcounter{tcbthm}Conjecture~\thetcbthm\if\relax\detokenize{#1}\relax\else\ (#1)\fi},
}

% Exemple (barre verte)
\newtcolorbox{exemple}[1][]{
    mytheoremstyle,
    borderline west={2pt}{0pt}{examplecolor},
    coltitle=examplecolor,
    title={\refstepcounter{tcbthm}Exemple~\thetcbthm\if\relax\detokenize{#1}\relax\else\ (#1)\fi},
}

% Exercice (barre orange vif)
\newtcolorbox{exercice}[1][]{
    mytheoremstyle,
    borderline west={2pt}{0pt}{exercisecolor},
    coltitle=exercisecolor,
    title={\refstepcounter{tcbthm}Exercice~\thetcbthm\if\relax\detokenize{#1}\relax\else\ (#1)\fi},
}

% Remarque numérotée (barre grise claire, texte en italique)
\newtcolorbox{remarque}[1][]{
    mytheoremstyle,
    borderline west={2pt}{0pt}{black!40},
    coltitle=black!60,
    fontupper=\itshape,
    title={\refstepcounter{tcbthm}Remarque~\thetcbthm\if\relax\detokenize{#1}\relax\else\ (#1)\fi},
}

% ========================================================================
% ENVIRONNEMENTS SANS NUMÉROTATION
% ========================================================================

% Remarque simple sans numérotation et sans barre latérale
\newenvironment{remarquesimp}[1][]%
{%
    \par\medskip\noindent%
    \textbf{\sffamily\color{black!60}Remarque%
    \if\relax\detokenize{#1}\relax\else\ (#1)\fi.}%
    \itshape\space%
}%
{%
    \par\medskip%
}

% Notation
\newenvironment{notation}[1][]%
{%
    \par\medskip\noindent%
    \textbf{\sffamily\color{black!60}Notation%
    \if\relax\detokenize{#1}\relax\else\ (#1)\fi :}%
    \itshape\space%
}%
{%
    \par\medskip%
}

% ========================================================================
% CONFIGURATION DES EN-TÊTES ET PIEDS DE PAGE
% ========================================================================

% Activer le style scrheadings
\pagestyle{scrheadings}

% Nettoyer les styles par défaut
\clearpairofpagestyles

% Configuration des en-têtes pour pages paires/impaires
\ihead{\small\leftmark}  % En-tête intérieur : nom du chapitre
\ohead{\textcolor{primarycolor}{\thepage}}  % En-tête extérieur : numéro de page

% Pied de page pour le style plain (début de chapitre)
\cfoot[\textcolor{primarycolor}{\thepage}]{}

% Ligne de séparation colorée
\ModifyLayer[addvoffset=-.6ex]{scrheadings.foot.above.line}% un peu plus haut
\ModifyLayer[addvoffset=.6ex]{plain.scrheadings.foot.above.line}% un peu plus haut  
\setkomafont{pageheadfoot}{\normalfont\normalcolor}
\setkomafont{pagination}{\normalfont\normalcolor}
\KOMAoptions{headsepline=0.5pt:head}
\setkomafont{headsepline}{\color{primarycolor}}

% ========================================================================
% PERSONNALISATION DES TITRES DE SECTIONS (KOMA)
% ========================================================================

\definecolor{partgray}{gray}{0.85}
\definecolor{parttitle}{HTML}{222222}

% === PARTIES ===
\setkomafont{part}{\color{parttitle}\Huge\bfseries}

\renewcommand*\partformat{}
\renewcommand*\partheadstartvskip{\vspace*{3cm}}
\renewcommand*\partheadmidvskip{1em}
\renewcommand*\partheadendvskip{\vspace*{5cm}}

\renewcommand*\partlineswithprefixformat[3]{%
  \begin{tikzpicture}[remember picture,overlay]
    \node[opacity=0.15,scale=30,text=partgray] at (current page.center)
      {\thepart};
  \end{tikzpicture}%
  \begin{center}
    {\Large\textsc{Partie~\thepart}}\\[1em]
    {\Huge\bfseries\color{parttitle} #3}
  \end{center}
}

\RedeclareSectionCommand[
  beforeskip=0pt,
  afterskip=5cm
]{part}

% === CHAPITRES ===
\setkomafont{chapter}{\normalfont\huge\bfseries\color{primarycolor}}
\setkomafont{chapterprefix}{\normalfont\Large\color{primarycolor}}

\renewcommand*{\chapterformat}{%
  \chapappifchapterprefix{\ }%
  \thechapter\autodot\enskip%
}

\RedeclareSectionCommand[
  beforeskip=0pt,
  afterskip=20pt
]{chapter}

% === SECTIONS ===
\setkomafont{section}{\normalfont\Large\bfseries\color{primarycolor}}

\RedeclareSectionCommand[
  beforeskip=-3.5ex plus -1ex minus -0.2ex,
  afterskip=2.3ex plus 0.2ex
]{section}

% === SOUS-SECTIONS ===
\setkomafont{subsection}{\normalfont\large\bfseries\color{accentcolor}}

\RedeclareSectionCommand[
  beforeskip=-3.25ex plus -1ex minus -0.2ex,
  afterskip=1.5ex plus 0.2ex
]{subsection}

% === SOUS-SOUS-SECTIONS ===
\setkomafont{subsubsection}{\normalfont\normalsize\bfseries\color{accentcolor!80}}

% ========================================================================
% COMMANDES MATHÉMATIQUES PERSONNALISÉES
% ========================================================================

% Ensembles classiques
\newcommand{\N}{\mathbb{N}}
\newcommand{\Z}{\mathbb{Z}}
\newcommand{\Q}{\mathbb{Q}}
\newcommand{\R}{\mathbb{R}}
\newcommand{\C}{\mathbb{C}}
\newcommand{\K}{\mathbb{K}}
\newcommand{\partie}{\mathcal{P}}  % Ensemble des parties

% Autres notations utiles
\DeclareMathOperator{\Card}{Card}
\DeclareMathOperator{\im}{Im}
\DeclareMathOperator{\dom}{dom}
\DeclareMathOperator{\Ker}{Ker}
\DeclareMathOperator{\Hom}{Hom}
\DeclareMathOperator{\End}{End}
\DeclareMathOperator{\Aut}{Aut}

% ========================================================================
% MÉTADONNÉES DU DOCUMENT
% ========================================================================
\title{Recueil Mathématique}
\subtitle{Un voyage à travers les mathématiques}
\author{Tiago Piette}
\date{2025}
\newcommand{\version}{v1.0}