\begin{titlepage}
\thispagestyle{empty}

% ======== Palette professionnelle et harmonieuse ========
\definecolor{primarycolor}{RGB}{0,61,102}      % Bleu pétrole profond
\definecolor{secondarycolor}{RGB}{0,139,139}   % Vert sarcelle élégant
\definecolor{accentcolor}{RGB}{255,180,60}     % Doré doux et chaud
% =========================================================

\begin{tikzpicture}[remember picture,overlay]
    % Fond coloré
    \fill[primarycolor!10] (current page.south west) rectangle (current page.north east);
    
    % Motifs géométriques décoratifs
    \foreach \i in {1,...,20}{
        \pgfmathsetmacro{\opacity}{0.1+0.02*\i}
        \fill[primarycolor, opacity=\opacity] 
            ($(current page.north west)+(0,-\i)$) -- 
            ($(current page.north west)+(\i,-\i)$) --
            ($(current page.north west)+(\i,0)$) -- cycle;
    }
    
    \foreach \i in {1,...,20}{
        \pgfmathsetmacro{\opacity}{0.1+0.02*\i}
        \fill[secondarycolor, opacity=\opacity] 
            ($(current page.south east)+(0,\i)$) -- 
            ($(current page.south east)+(-\i,\i)$) --
            ($(current page.south east)+(-\i,0)$) -- cycle;
    }
    
    % Boîte centrale pour le titre
    \node[
        rectangle,
        draw=primarycolor,
        line width=2pt,
        inner sep=20pt,
        fill=white,
        drop shadow={shadow xshift=5pt, shadow yshift=-5pt, opacity=0.3}
    ] at (current page.center) {
        \begin{minipage}{0.8\textwidth}
            \centering
            
            % Ornement supérieur
            {\color{primarycolor}\rule{0.7\linewidth}{1.5pt}}
            
            \vspace{1cm}
            
            % Titre principal
            {\Huge\bfseries\color{primarycolor} Recueil Mathématique\par}
            
            \vspace{0.5cm}
            
            % Sous-titre
            {\LARGE\itshape\color{secondarycolor} Un voyage à travers les mathématiques \par}
            
            \vspace{1.5cm}
            
            % Ornement central
            \begin{center}
            \begin{tikzpicture}
                \draw[primarycolor, line width=1pt] (0,0) circle (0.8cm);
                \draw[secondarycolor, line width=1pt] (0,0) circle (0.6cm);
                \foreach \angle in {0,60,...,300} {
                    \draw[accentcolor, line width=0.5pt] (0,0) -- (\angle:0.8cm);
                }
            \end{tikzpicture}
            \end{center}
            
            \vspace{1.5cm}
            
            % Auteur
            {\Large\scshape\color{black} Tiago Piette \par}
            
            \vspace{0.5cm}
            
            % Date et version
            {\large\color{black!70} 2025 \par}
            {\small\color{black!50} Version: \version\par}
            
            \vspace{1cm}
            
            % Ornement inférieur
            {\color{primarycolor}\rule{0.7\linewidth}{1.5pt}}
            
        \end{minipage}
    };
    
    % Symboles mathématiques discrets
    \node[opacity=0.08, scale=6, color=white]
        at ($(current page.center)+(-7,7)$) {$G \rtimes H$};
    \node[opacity=0.08, scale=6, color=white]
        at ($(current page.center)+(7,7)$) {$\phi: G \to H$};
    \node[opacity=0.10, scale=6, color=white]
        at ($(current page.center)+(-7,-7.5)$) {$S_n$};
    \node[opacity=0.10, scale=6, color=white]
        at ($(current page.center)+(7,-7.5)$) {$G \times X$};
    
\end{tikzpicture}

\end{titlepage}
