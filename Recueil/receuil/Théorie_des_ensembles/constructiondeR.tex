\chapter{Construction de $\R$}

\section{Motivation : Les lacunes de $\Q$}

L'ensemble $\Q$ des nombres rationnels est remarquable à bien des égards : il est dense (entre deux rationnels distincts, il en existe toujours un autre), on peut y effectuer toutes les opérations arithmétiques usuelles (addition, soustraction, multiplication, division par un non-nul), et il possède un ordre total compatible avec ces opérations. Pourtant, malgré ces qualités, $\Q$ souffre de graves lacunes qui le rendent inadéquat pour l'analyse mathématique.

\subsection{Première lacune : Des équations sans solution}

\begin{theoreme}[L'irrationalité de $\sqrt{2}$]
Il n'existe aucun nombre rationnel $x$ tel que $x^2 = 2$.
\end{theoreme}

\begin{proof}
Procédons par l'absurde. Supposons qu'il existe $x \in \Q$ tel que $x^2 = 2$. On peut écrire $x = \frac{a}{b}$ où $a, b \in \Z$ avec $b \neq 0$. Quitte à simplifier par les facteurs communs, on peut supposer que la fraction est irréductible, c'est-à-dire que $a$ et $b$ n'ont pas de facteur commun autre que $\pm 1$.

De l'égalité $x^2 = 2$, on obtient :
\[
\left(\frac{a}{b}\right)^2 = 2 \quad \Rightarrow \quad \frac{a^2}{b^2} = 2 \quad \Rightarrow \quad a^2 = 2b^2
\]

Cette égalité montre que $a^2$ est pair. Or, si le carré d'un entier est pair, alors cet entier lui-même est pair (en effet, le carré d'un impair est impair : $(2k+1)^2 = 4k^2 + 4k + 1 = 2(2k^2 + 2k) + 1$). Donc $a$ est pair, c'est-à-dire qu'il existe $k \in \Z$ tel que $a = 2k$.

En substituant dans l'égalité $a^2 = 2b^2$ :
\[
(2k)^2 = 2b^2 \quad \Rightarrow \quad 4k^2 = 2b^2 \quad \Rightarrow \quad 2k^2 = b^2
\]

Donc $b^2$ est pair, donc $b$ est pair.

Ainsi, $a$ et $b$ sont tous les deux pairs, donc divisibles par $2$. Ceci contredit l'hypothèse que la fraction $\frac{a}{b}$ est irréductible.

Cette contradiction provient de l'hypothèse initiale : il n'existe donc pas de rationnel $x$ tel que $x^2 = 2$.
\end{proof}

\begin{remarque}
Ce résultat est historiquement important : il fut découvert par l'école pythagoricienne dans l'Antiquité grecque et constitua une crise majeure pour leur philosophie mathématique. Il montre qu'il existe des grandeurs géométriques (la diagonale d'un carré de côté 1) qui ne peuvent pas s'exprimer comme rapports d'entiers.

De manière similaire, on peut montrer que $\sqrt{3}$, $\sqrt{5}$, $\sqrt[3]{2}$, et bien d'autres nombres ne sont pas rationnels.
\end{remarque}

\subsection{Deuxième lacune : Des suites sans limite}

Considérons la suite $(u_n)$ définie par récurrence :
\[
u_0 = 1, \quad u_{n+1} = \frac{u_n + \frac{2}{u_n}}{2}
\]

Cette suite a des propriétés remarquables :

\begin{proposition}
La suite $(u_n)$ définie ci-dessus satisfait :
\begin{enumerate}
    \item $u_n > 0$ pour tout $n \in \N$
    \item $u_n \geq \sqrt{2}$ pour tout $n \geq 1$ (au sens où si $\sqrt{2}$ existait)
    \item $(u_n)$ est décroissante à partir du rang 1
    \item Les termes de la suite se rapprochent arbitrairement : $|u_{n+1} - u_n|$ devient aussi petit qu'on veut
\end{enumerate}
\end{proposition}

\begin{proof}[Preuve informelle]
Si la suite convergeait vers une limite $L$, en passant à la limite dans la relation de récurrence :
\[
L = \frac{L + \frac{2}{L}}{2} \quad \Rightarrow \quad 2L = L + \frac{2}{L} \quad \Rightarrow \quad L = \frac{2}{L} \quad \Rightarrow \quad L^2 = 2
\]

Donc si cette suite converge, sa limite vérifie $L^2 = 2$. Mais nous venons de montrer qu'aucun rationnel ne vérifie cette équation ! Ainsi, cette suite de nombres rationnels ne peut converger vers aucun nombre rationnel.
\end{proof}

\begin{remarque}
Cette situation est paradoxale : intuitivement, la suite "devrait" converger car ses termes se rapprochent de plus en plus. Il y a un "trou" dans $\Q$ là où devrait se trouver $\sqrt{2}$. La construction de $\R$ aura précisément pour but de "combler" tous ces trous.
\end{remarque}

\subsection{Troisième lacune : L'incomplétude de $\Q$}

\begin{definition}[Intuitive]
Une suite $(u_n)$ est dite \textbf{convergente} si ses termes se rapprochent arbitrairement d'une certaine valeur (appelée limite).

Une suite $(u_n)$ satisfait la \textbf{condition de Cauchy} si ses termes se rapprochent arbitrairement les uns des autres, c'est-à-dire si pour tout degré de précision souhaité, à partir d'un certain rang, tous les termes sont aussi proches qu'on veut.
\end{definition}

Dans un ensemble "complet", toute suite satisfaisant la condition de Cauchy devrait converger. Or, $\Q$ n'est pas complet : il existe des suites de rationnels qui satisfont la condition de Cauchy mais ne convergent vers aucun rationnel (comme la suite qui "tend vers $\sqrt{2}$").

L'objectif de ce chapitre est de construire $\R$, le \textbf{complété} de $\Q$, dans lequel toute suite de Cauchy convergera.

\section{La valeur absolue et la distance}

Avant de formaliser la notion de suite de Cauchy, introduisons un outil essentiel : la valeur absolue, qui mesure la "distance" entre deux nombres.

\begin{definition}[Valeur absolue]
Pour $x \in \Q$, on définit la \textbf{valeur absolue} de $x$ par :
\[
|x| := \begin{cases}
x & \text{si } x \geq 0\\
-x & \text{si } x < 0
\end{cases}
\]
\end{definition}

\begin{remarque}
Géométriquement, $|x|$ représente la distance entre $x$ et $0$ sur la droite des nombres. De même, $|x - y|$ représente la distance entre $x$ et $y$.
\end{remarque}

\begin{proposition}[Propriétés de la valeur absolue]
Pour tous $x, y \in \Q$ :
\begin{enumerate}
    \item $|x| \geq 0$ et $|x| = 0 \Leftrightarrow x = 0$
    \item $|-x| = |x|$
    \item $|xy| = |x| \cdot |y|$
    \item $|x + y| \leq |x| + |y|$ (inégalité triangulaire)
\end{enumerate}
\end{proposition}

\begin{proof}
Les trois premières propriétés se vérifient par disjonction de cas.

Pour l'inégalité triangulaire, distinguons les cas selon les signes de $x$ et $y$. Si $x, y \geq 0$ :
\[
|x + y| = x + y = |x| + |y|
\]

Si $x \geq 0$ et $y < 0$, alors soit $x + y \geq 0$ auquel cas :
\[
|x + y| = x + y \leq x = x + y - y \leq x - y = x + |y| = |x| + |y|
\]
soit $x + y < 0$ auquel cas un raisonnement similaire s'applique.

Les autres cas se traitent de manière analogue.
\end{proof}

\begin{remarque}
L'inégalité triangulaire est fondamentale : elle exprime que "le chemin direct est le plus court". Géométriquement, la distance de $0$ à $x+y$ est au plus la somme des distances de $0$ à $x$ et de $x$ à $x+y$.
\end{remarque}

\section{Les suites de Cauchy}

\subsection{Définition et premières propriétés}

\begin{definition}[Suite de Cauchy]
Une suite $(u_n)_{n \in \N}$ d'éléments de $\Q$ est dite \textbf{de Cauchy} si :
\[
\forall \varepsilon \in \Q_{+}^*, \; \exists N \in \N, \; \forall m, n \geq N : |u_m - u_n| < \varepsilon
\]
\end{definition}

\begin{remarque}
Autrement dit, une suite est de Cauchy si, pour tout degré de précision $\varepsilon > 0$, à partir d'un certain rang $N$, tous les termes de la suite sont à une distance mutuelle inférieure à $\varepsilon$. Les termes de la suite se "resserrent" de plus en plus.

Ceci est une condition nécessaire à la convergence : si une suite converge vers une limite $L$, alors pour $n, m$ assez grands, $u_n$ et $u_m$ sont tous deux proches de $L$, donc proches l'un de l'autre. Dans $\R$, cette condition sera également suffisante (c'est la propriété de complétude).
\end{remarque}

\begin{exemple}
La suite $u_n = \frac{1}{n}$ est de Cauchy.

En effet, pour $m > n$, on a :
\[
\left|u_m - u_n\right| = \left|\frac{1}{m} - \frac{1}{n}\right| = \frac{n - m}{mn} \leq \frac{n}{mn} = \frac{1}{m} \leq \frac{1}{n}
\]

Donc pour tout $\varepsilon > 0$, si on choisit $N > \frac{1}{\varepsilon}$, alors pour tous $m, n \geq N$ :
\[
|u_m - u_n| \leq \frac{1}{n} \leq \frac{1}{N} < \varepsilon
\]
\end{exemple}

\begin{exemple}
La suite $u_0 = 1$, $u_{n+1} = \frac{u_n + 2/u_n}{2}$ est de Cauchy dans $\Q$, bien qu'elle ne converge vers aucun rationnel.
\end{exemple}

\begin{notation}
On note $\mathcal{C}(\Q)$ l'ensemble de toutes les suites de Cauchy à valeurs dans $\Q$ :
\[
\mathcal{C}(\Q) := \{(u_n)_{n \in \N} \mid u_n \in \Q \text{ pour tout } n, \text{ et } (u_n) \text{ est de Cauchy}\}
\]
\end{notation}

\begin{lemme}[Les suites de Cauchy sont bornées]\label{lem:cauchy_bornee}
Toute suite de Cauchy dans $\Q$ est bornée.
\end{lemme}

\begin{proof}
Soit $(u_n)$ une suite de Cauchy. Appliquons la définition avec $\varepsilon = 1$ : il existe $N \in \N$ tel que pour tous $m, n \geq N$ :
\[
|u_m - u_n| < 1
\]

En particulier, pour tout $n \geq N$ :
\[
|u_n - u_N| < 1 \quad \Rightarrow \quad |u_n| \leq |u_N| + |u_n - u_N| < |u_N| + 1
\]

Posons :
\[
M := \max\{|u_0|, |u_1|, \ldots, |u_{N-1}|, |u_N| + 1\}
\]

Alors pour tout $n \in \N$, on a $|u_n| \leq M$, ce qui signifie que la suite est bornée.
\end{proof}

\subsection{Opérations sur les suites de Cauchy}

\begin{proposition}[Stabilité par addition]
Si $(u_n)$ et $(v_n)$ sont des suites de Cauchy, alors $(u_n + v_n)$ est une suite de Cauchy.
\end{proposition}

\begin{proof}
Soit $\varepsilon > 0$. Puisque $(u_n)$ est de Cauchy, il existe $N_1$ tel que pour tous $m, n \geq N_1$ :
\[
|u_m - u_n| < \frac{\varepsilon}{2}
\]

De même, il existe $N_2$ tel que pour tous $m, n \geq N_2$ :
\[
|v_m - v_n| < \frac{\varepsilon}{2}
\]

Pour $m, n \geq \max(N_1, N_2)$, l'inégalité triangulaire donne :
\[
|(u_m + v_m) - (u_n + v_n)| = |(u_m - u_n) + (v_m - v_n)| \leq |u_m - u_n| + |v_m - v_n| < \frac{\varepsilon}{2} + \frac{\varepsilon}{2} = \varepsilon
\]

Donc $(u_n + v_n)$ est de Cauchy.
\end{proof}

\begin{proposition}[Stabilité par multiplication]
Si $(u_n)$ et $(v_n)$ sont des suites de Cauchy, alors $(u_n \cdot v_n)$ est une suite de Cauchy.
\end{proposition}

\begin{proof}
Par le lemme \ref{lem:cauchy_bornee}, il existe $M, K > 0$ tels que $|u_n| \leq M$ et $|v_n| \leq K$ pour tous $n$.

Soit $\varepsilon > 0$. Il existe $N_1$ tel que pour tous $m, n \geq N_1$ : $|u_m - u_n| < \frac{\varepsilon}{2K}$.
Il existe $N_2$ tel que pour tous $m, n \geq N_2$ : $|v_m - v_n| < \frac{\varepsilon}{2M}$.

Pour $m, n \geq \max(N_1, N_2)$, on utilise l'astuce :
\[
u_m v_m - u_n v_n = u_m v_m - u_m v_n + u_m v_n - u_n v_n = u_m(v_m - v_n) + v_n(u_m - u_n)
\]

Donc :
\begin{align*}
|u_m v_m - u_n v_n| &\leq |u_m| \cdot |v_m - v_n| + |v_n| \cdot |u_m - u_n|\\
&\leq M \cdot \frac{\varepsilon}{2M} + K \cdot \frac{\varepsilon}{2K}\\
&= \frac{\varepsilon}{2} + \frac{\varepsilon}{2} = \varepsilon
\end{align*}

Donc $(u_n v_n)$ est de Cauchy.
\end{proof}

\section{La relation d'équivalence}

Deux suites de Cauchy peuvent "tendre vers le même nombre" sans être égales terme à terme. Par exemple, les suites $u_n = \frac{1}{n}$ et $v_n = \frac{2}{n}$ sont différentes, mais elles "tendent toutes deux vers 0". Nous voulons identifier de telles suites.

\begin{definition}[Relation d'équivalence]
On définit une relation binaire $\sim$ sur $\mathcal{C}(\Q)$ par :
\[
(u_n) \sim (v_n) \quad :\Leftrightarrow \quad \lim_{n \to \infty} |u_n - v_n| = 0
\]

Plus précisément, $(u_n) \sim (v_n)$ si et seulement si :
\[
\forall \varepsilon \in \Q_{+}^*, \; \exists N \in \N, \; \forall n \geq N : |u_n - v_n| < \varepsilon
\]
\end{definition}

\begin{remarque}
Intuitivement, deux suites sont équivalentes si leur différence tend vers zéro, c'est-à-dire si elles "convergent vers le même nombre" (même si ce nombre n'existe pas nécessairement dans $\Q$).
\end{remarque}

\begin{exemple}
\begin{itemize}
    \item Les suites $u_n = \frac{1}{n}$ et $v_n = \frac{2}{n}$ sont équivalentes car $|u_n - v_n| = \frac{1}{n} \to 0$.
    \item Les suites constantes $u_n = 3$ et $v_n = 3$ sont équivalentes (évidemment).
    \item La suite $u_0 = 1$, $u_{n+1} = \frac{u_n + 2/u_n}{2}$ est équivalente à toute autre suite qui "converge vers $\sqrt{2}$".
\end{itemize}
\end{exemple}

\begin{theoreme}
La relation $\sim$ est une relation d'équivalence sur $\mathcal{C}(\Q)$.
\end{theoreme}

\begin{proof}
Vérifions les trois propriétés caractéristiques.

\textbf{Réflexivité.} Pour toute suite $(u_n)$, on a $|u_n - u_n| = 0 < \varepsilon$ pour tout $\varepsilon > 0$, donc $(u_n) \sim (u_n)$.

\textbf{Symétrie.} Si $(u_n) \sim (v_n)$, alors $|u_n - v_n| \to 0$. Or $|v_n - u_n| = |u_n - v_n|$, donc $|v_n - u_n| \to 0$, donc $(v_n) \sim (u_n)$.

\textbf{Transitivité.} Supposons $(u_n) \sim (v_n)$ et $(v_n) \sim (w_n)$. Soit $\varepsilon > 0$. Il existe $N_1$ tel que pour $n \geq N_1$ : $|u_n - v_n| < \frac{\varepsilon}{2}$. Il existe $N_2$ tel que pour $n \geq N_2$ : $|v_n - w_n| < \frac{\varepsilon}{2}$.

Pour $n \geq \max(N_1, N_2)$, par l'inégalité triangulaire :
\[
|u_n - w_n| \leq |u_n - v_n| + |v_n - w_n| < \frac{\varepsilon}{2} + \frac{\varepsilon}{2} = \varepsilon
\]

Donc $(u_n) \sim (w_n)$.
\end{proof}

\begin{definition}[Classe d'équivalence]
Pour une suite de Cauchy $(u_n)$, on note $[(u_n)]$ sa classe d'équivalence pour la relation $\sim$ :
\[
[(u_n)] := \{(v_n) \in \mathcal{C}(\Q) \mid (v_n) \sim (u_n)\}
\]
\end{definition}

\begin{remarque}
Une classe d'équivalence $[(u_n)]$ représente "le nombre vers lequel tend la suite $(u_n)$", même si ce nombre n'existe pas dans $\Q$. Par exemple :
\begin{itemize}
    \item La classe de $(0, 0, 0, \ldots)$ représentera le nombre réel $0$
    \item La classe de $(2, 2, 2, \ldots)$ représentera le nombre réel $2$
    \item La classe de la suite qui "tend vers $\sqrt{2}$" représentera le nombre réel $\sqrt{2}$
    \item La classe de $(3, 3.1, 3.14, 3.141, 3.1415, \ldots)$ représentera le nombre réel $\pi$
\end{itemize}
\end{remarque}

\section{Définition de $\R$}

Nous sommes maintenant prêts à définir l'ensemble des nombres réels.

\begin{definition}[Les nombres réels]
On définit l'ensemble des \textbf{nombres réels} par :
\[
\R := \mathcal{C}(\Q) / \sim \; = \; \{[(u_n)] \mid (u_n) \in \mathcal{C}(\Q)\}
\]
\end{definition}

\begin{remarque}
L'axiome de remplacement de la théorie ZF garantit l'existence de cet ensemble quotient. Chaque nombre réel est une classe d'équivalence de suites de Cauchy de rationnels. Cette construction s'appelle la **complétion de $\Q$ par les suites de Cauchy**, ou parfois la **construction de Cantor**.

Il existe une autre construction équivalente, due à Dedekind, basée sur les "coupures" de $\Q$. Les deux constructions produisent des ensembles isomorphes.
\end{remarque}

\subsection{Plongement de $\Q$ dans $\R$}

\begin{definition}
Pour $q \in \Q$, on note $\bar{q}$ la suite constante définie par $\bar{q}_n = q$ pour tout $n \in \N$.

On définit l'application $\iota : \Q \to \R$ par :
\[
\iota(q) := [(\bar{q})] = [(q, q, q, \ldots)]
\]
\end{definition}

\begin{proposition}
\begin{enumerate}
    \item Pour tout $q \in \Q$, la suite constante $\bar{q}$ est de Cauchy
    \item L'application $\iota$ est injective
\end{enumerate}
\end{proposition}

\begin{proof}
(1) Pour la suite $\bar{q}$, on a $|\bar{q}_m - \bar{q}_n| = |q - q| = 0 < \varepsilon$ pour tous $m, n$, donc elle est de Cauchy.

(2) Soient $p, q \in \Q$ tels que $\iota(p) = \iota(q)$, c'est-à-dire $[(\bar{p})] = [(\bar{q})]$. Alors $(\bar{p}) \sim (\bar{q})$, donc $|\bar{p}_n - \bar{q}_n| = |p - q| \to 0$.

Mais $|p - q|$ est une constante (elle ne dépend pas de $n$). Si cette constante tend vers $0$, elle doit être égale à $0$, donc $p = q$.
\end{proof}

\begin{remarque}
Cette injection permet d'identifier $\Q$ avec un sous-ensemble de $\R$. Désormais, par abus de notation, nous écrirons souvent $q$ au lieu de $[(\bar{q})]$ pour $q \in \Q$, considérant ainsi $\Q \subseteq \R$. Cette identification respectera toutes les structures (addition, multiplication, ordre) comme nous le verrons.
\end{remarque}

\section{Opérations sur $\R$}

\subsection{L'opposé}

\begin{definition}
Pour $x = [(u_n)] \in \R$, on définit l'\textbf{opposé} de $x$ par :
\[
-x := [(-u_n)]
\]
où $(-u_n)$ désigne la suite définie par $(-u_n)_k = -u_k$ pour tout $k$.
\end{definition}

\begin{lemme}[Bonne définition de l'opposé]
\begin{enumerate}
    \item Si $(u_n)$ est de Cauchy, alors $(-u_n)$ est de Cauchy
    \item Si $(u_n) \sim (v_n)$, alors $(-u_n) \sim (-v_n)$
\end{enumerate}
\end{lemme}

\begin{proof}
(1) Soit $\varepsilon > 0$. Puisque $(u_n)$ est de Cauchy, il existe $N$ tel que pour tous $m, n \geq N$ : $|u_m - u_n| < \varepsilon$.

Alors pour $m, n \geq N$ :
\[
|(-u_m) - (-u_n)| = |-(u_m - u_n)| = |u_m - u_n| < \varepsilon
\]

Donc $(-u_n)$ est de Cauchy.

(2) Si $(u_n) \sim (v_n)$, alors $|u_n - v_n| \to 0$. Or :
\[
|(-u_n) - (-v_n)| = |-(u_n - v_n)| = |u_n - v_n| \to 0
\]

Donc $(-u_n) \sim (-v_n)$.
\end{proof}

\begin{proposition}
Pour tout $x \in \R$ : $-(-x) = x$.
\end{proposition}

\begin{proof}
Si $x = [(u_n)]$, alors :
\[
-x = [(-u_n)] \quad \text{et} \quad -(-x) = [(-(-u_n))] = [(u_n)] = x
\]
car $-(-u_k) = u_k$ pour tout $k$.
\end{proof}

\subsection{L'addition}

\begin{definition}
Pour $x = [(u_n)]$ et $y = [(v_n)]$ dans $\R$, on définit :
\[
x + y := [(u_n + v_n)]
\]
où $(u_n + v_n)$ désigne la suite définie par $(u_n + v_n)_k = u_k + v_k$ pour tout $k$.
\end{definition}

\begin{lemme}[Bonne définition de l'addition]
\begin{enumerate}
    \item Si $(u_n)$ et $(v_n)$ sont de Cauchy, alors $(u_n + v_n)$ est de Cauchy
    \item Si $(u_n) \sim (u'_n)$ et $(v_n) \sim (v'_n)$, alors $(u_n + v_n) \sim (u'_n + v'_n)$
\end{enumerate}
\end{lemme}

\begin{proof}
Le point (1) a été démontré précédemment (stabilité par addition).

(2) Supposons $|u_n - u'_n| \to 0$ et $|v_n - v'_n| \to 0$. Alors :
\[
|(u_n + v_n) - (u'_n + v'_n)| = |(u_n - u'_n) + (v_n - v'_n)| \leq |u_n - u'_n| + |v_n - v'_n| \to 0
\]

Donc $(u_n + v_n) \sim (u'_n + v'_n)$.
\end{proof}

\begin{definition}
On note $0$ ou $0_\R$ l'élément $[(\bar{0})] = [(0, 0, 0, \ldots)] \in \R$.
\end{definition}

\begin{theoreme}[Propriétés de l'addition]
L'addition sur $\R$ satisfait les propriétés suivantes :
\begin{enumerate}
    \item \textbf{Associativité :} $\forall x, y, z \in \R, \; (x + y) + z = x + (y + z)$
    \item \textbf{Élément neutre :} $\forall x \in \R, \; 0 + x = x = x + 0$
    \item \textbf{Opposé :} $\forall x \in \R, \; x + (-x) = 0 = (-x) + x$
    \item \textbf{Commutativité :} $\forall x, y \in \R, \; x + y = y + x$
\end{enumerate}
\end{theoreme}

\begin{proof}
Toutes ces propriétés se vérifient en utilisant les propriétés correspondantes dans $\Q$, appliquées terme à terme.

\textbf{Associativité.} Soient $x = [(u_n)]$, $y = [(v_n)]$, $z = [(w_n)]$. Alors :
\begin{align*}
(x + y) + z &= [(u_n + v_n)] + [(w_n)] = [((u_n + v_n) + w_n)]\\
&= [(u_n + (v_n + w_n))] = [(u_n)] + [(v_n + w_n)]\\
&= x + (y + z)
\end{align*}
par associativité de l'addition dans $\Q$.

\textbf{Neutralité.} Pour $x = [(u_n)]$ :
\[
0 + x = [(\bar{0})] + [(u_n)] = [(0 + u_n)] = [(u_n)] = x
\]
car $0 + u_k = u_k$ dans $\Q$ pour tout $k$.

\textbf{Opposé.} Pour $x = [(u_n)]$ :
\[
x + (-x) = [(u_n)] + [(-u_n)] = [(u_n + (-u_n))] = [(0, 0, 0, \ldots)] = 0
\]
car $u_k + (-u_k) = 0$ dans $\Q$ pour tout $k$.

\textbf{Commutativité.} Pour $x = [(u_n)]$, $y = [(v_n)]$ :
\[
x + y = [(u_n + v_n)] = [(v_n + u_n)] = y + x
\]
par commutativité de l'addition dans $\Q$.
\end{proof}

\begin{proposition}[Compatibilité du plongement avec l'addition]
Pour tous $p, q \in \Q$ : $\iota(p + q) = \iota(p) + \iota(q)$.
\end{proposition}

\begin{proof}
On a :
\[
\iota(p) + \iota(q) = [(\bar{p})] + [(\bar{q})] = [(\bar{p} + \bar{q})] = [(\overline{p+q})] = \iota(p+q)
\]
car $\bar{p}_k + \bar{q}_k = p + q = \overline{(p+q)}_k$ pour tout $k$.
\end{proof}

\subsection{La multiplication}

\begin{definition}
Pour $x = [(u_n)]$ et $y = [(v_n)]$ dans $\R$, on définit :
\[
x \cdot y := [(u_n \cdot v_n)]
\]
où $(u_n \cdot v_n)$ désigne la suite définie par $(u_n \cdot v_n)_k = u_k \cdot v_k$ pour tout $k$.
\end{definition}

\begin{lemme}[Bonne définition de la multiplication]
\begin{enumerate}
    \item Si $(u_n)$ et $(v_n)$ sont de Cauchy, alors $(u_n \cdot v_n)$ est de Cauchy
    \item Si $(u_n) \sim (u'_n)$ et $(v_n) \sim (v'_n)$, alors $(u_n \cdot v_n) \sim (u'_n \cdot v'_n)$
\end{enumerate}
\end{lemme}

\begin{proof}
Le point (1) a été démontré précédemment (stabilité par multiplication).

(2) Les suites $(u_n)$, $(u'_n)$, $(v_n)$, $(v'_n)$ sont bornées (disons par $M$). Si $|u_n - u'_n| \to 0$ et $|v_n - v'_n| \to 0$, alors :
\begin{align*}
|u_n v_n - u'_n v'_n| &= |u_n v_n - u_n v'_n + u_n v'_n - u'_n v'_n|\\
&\leq |u_n||v_n - v'_n| + |v'_n||u_n - u'_n|\\
&\leq M|v_n - v'_n| + M|u_n - u'_n| \to 0
\end{align*}

Donc $(u_n v_n) \sim (u'_n v'_n)$.
\end{proof}

\begin{definition}
On note $1$ ou $1_\R$ l'élément $[(\bar{1})] = [(1, 1, 1, \ldots)] \in \R$.
\end{definition}

\begin{theoreme}[Propriétés de la multiplication]
La multiplication sur $\R$ satisfait les propriétés suivantes :
\begin{enumerate}
    \item \textbf{Associativité :} $\forall x, y, z \in \R, \; (x \cdot y) \cdot z = x \cdot (y \cdot z)$
    \item \textbf{Commutativité :} $\forall x, y \in \R, \; x \cdot y = y \cdot x$
    \item \textbf{Élément neutre :} $\forall x \in \R, \; 1 \cdot x = x = x \cdot 1$
    \item \textbf{Distributivité :} $\forall x, y, z \in \R, \; x \cdot (y + z) = x \cdot y + x \cdot z$
\end{enumerate}
\end{theoreme}

\begin{proof}
Similaire à celle pour l'addition, en utilisant les propriétés correspondantes dans $\Q$.
\end{proof}

\begin{theoreme}[Absence de diviseurs de zéro]
Pour tous $x, y \in \R$ : si $x \cdot y = 0$ alors $x = 0$ ou $y = 0$.
\end{theoreme}

\begin{proof}
Soient $x = [(u_n)]$ et $y = [(v_n)]$ tels que $x \cdot y = 0$. Alors $[(u_n v_n)] = [(0, 0, 0, \ldots)]$, donc $(u_n v_n) \sim (\bar{0})$, c'est-à-dire $|u_n v_n| \to 0$.

Supposons par l'absurde que $x \neq 0$ et $y \neq 0$. Alors il existe $\varepsilon, \delta > 0$ et $N_1, N_2$ tels que :
- Pour tout $n \geq N_1$ : $|u_n| \geq \varepsilon$
- Pour tout $n \geq N_2$ : $|v_n| \geq \delta$

Donc pour $n \geq \max(N_1, N_2)$ : $|u_n v_n| = |u_n||v_n| \geq \varepsilon \delta > 0$.

Ceci contredit $|u_n v_n| \to 0$. Donc $x = 0$ ou $y = 0$.
\end{proof}

\subsection{L'inverse}

La définition de l'inverse nécessite un travail préliminaire pour caractériser les réels non nuls.

\begin{lemme}[Caractérisation des réels non nuls]\label{lem:reel_non_nul}
Soit $x = [(u_n)] \in \R$. Les conditions suivantes sont équivalentes :
\begin{enumerate}
    \item $x \neq 0$
    \item Il existe $\varepsilon \in \Q_{+}^*$ et $N \in \N$ tels que pour tout $n \geq N$ : $|u_n| \geq \varepsilon$
\end{enumerate}
\end{lemme}

\begin{proof}
$(1 \Rightarrow 2)$ Si $x \neq 0$, alors $(u_n) \not\sim (\bar{0})$, donc il existe $\varepsilon > 0$ tel que pour une infinité d'indices $n$ : $|u_n - 0| = |u_n| \geq 3\varepsilon$.

Puisque $(u_n)$ est de Cauchy, il existe $N_0$ tel que pour tous $m, n \geq N_0$ : $|u_m - u_n| < \varepsilon$.

Soit $n_0 \geq N_0$ un indice tel que $|u_{n_0}| \geq 3\varepsilon$. Pour tout $n \geq N_0$ :
\[
|u_n| \geq |u_{n_0}| - |u_n - u_{n_0}| \geq 3\varepsilon - \varepsilon = 2\varepsilon
\]

En posant $N = N_0$ et en remplaçant $2\varepsilon$ par $\varepsilon$, on obtient le résultat.

$(2 \Rightarrow 1)$ Si $|u_n| \geq \varepsilon$ pour tout $n \geq N$, alors $|u_n - 0| \geq \varepsilon$ ne tend pas vers $0$, donc $(u_n) \not\sim (\bar{0})$, donc $x \neq 0$.
\end{proof}

\begin{definition}[Inverse]
Pour $x = [(u_n)] \in \R$ avec $x \neq 0$, on définit l'inverse de $x$ comme suit :

Soit $\varepsilon$ et $N$ donnés par le lemme \ref{lem:reel_non_nul}. On définit la suite $(w_n)$ par :
\[
w_n := \begin{cases}
\frac{1}{u_n} & \text{si } n \geq N\\
\frac{1}{\varepsilon} & \text{si } n < N
\end{cases}
\]

Alors on pose : $x^{-1} := [(w_n)]$.
\end{definition}

\begin{lemme}[Bonne définition de l'inverse]
\begin{enumerate}
    \item La suite $(w_n)$ est de Cauchy
    \item La définition ne dépend pas du choix de $N$ et $\varepsilon$
    \item Si $(u_n) \sim (u'_n)$ avec $x \neq 0$, alors les inverses correspondants sont équivalents
\end{enumerate}
\end{lemme}

\begin{proof}[Preuve du point (1)]
Pour $m, n \geq N$, on a :
\[
|w_m - w_n| = \left|\frac{1}{u_m} - \frac{1}{u_n}\right| = \frac{|u_n - u_m|}{|u_m||u_n|} \leq \frac{|u_n - u_m|}{\varepsilon^2}
\]

Puisque $(u_n)$ est de Cauchy, pour tout $\delta > 0$, il existe $N'$ tel que pour $m, n \geq N'$ : $|u_m - u_n| < \varepsilon^2 \delta$.

Donc pour $m, n \geq \max(N, N')$ : $|w_m - w_n| < \delta$, ce qui montre que $(w_n)$ est de Cauchy.

Les points (2) et (3) se démontrent par des calculs similaires.
\end{proof}

\begin{proposition}
Pour tout $x \in \R \setminus \{0\}$ :
\begin{enumerate}
    \item $x \cdot x^{-1} = 1 = x^{-1} \cdot x$
    \item $(x^{-1})^{-1} = x$
\end{enumerate}
\end{proposition}

\begin{proof}
(1) Si $x = [(u_n)]$ avec $x^{-1} = [(w_n)]$, alors pour $n \geq N$ : $u_n \cdot w_n = u_n \cdot \frac{1}{u_n} = 1$.

Donc $(u_n \cdot w_n)$ est égale à $1$ à partir du rang $N$, donc $(u_n \cdot w_n) \sim (\bar{1})$, donc $x \cdot x^{-1} = 1$.

(2) Immédiat par construction.
\end{proof}

\section{Ordre sur $\R$}

\subsection{Définition de l'ordre}

\begin{definition}
Pour $x = [(u_n)]$ et $y = [(v_n)]$ dans $\R$, on définit :
\[
x \leq y \quad :\Leftrightarrow \quad \exists N \in \N, \; \forall n \geq N : u_n \leq v_n
\]
\end{definition}

\begin{remarque}
Intuitivement, $x \leq y$ signifie qu'à partir d'un certain rang, tous les termes de la suite représentant $x$ sont inférieurs ou égaux à ceux de la suite représentant $y$.
\end{remarque}

\begin{lemme}[Bonne définition de l'ordre]
Si $(u_n) \sim (u'_n)$ et $(v_n) \sim (v'_n)$, alors :
\[
(\exists N, \forall n \geq N : u_n \leq v_n) \Leftrightarrow (\exists N', \forall n \geq N' : u'_n \leq v'_n)
\]
\end{lemme}

\begin{proof}[Preuve (esquisse)]
Supposons qu'il existe $N_0$ tel que pour $n \geq N_0$ : $u_n \leq v_n$.

Puisque $(u_n) \sim (u'_n)$, pour tout $\varepsilon > 0$, il existe $N_1$ tel que pour $n \geq N_1$ : $|u_n - u'_n| < \varepsilon$, donc $u'_n < u_n + \varepsilon$.

De même, il existe $N_2$ tel que pour $n \geq N_2$ : $v_n < v'_n + \varepsilon$.

Pour $n \geq \max(N_0, N_1, N_2)$ et pour $\varepsilon$ assez petit :
\[
u'_n < u_n + \varepsilon \leq v_n + \varepsilon < v'_n + 2\varepsilon
\]

On peut raffiner ce raisonnement pour obtenir $u'_n \leq v'_n$ éventuellement.
\end{proof}

\begin{definition}
On définit les relations dérivées :
\begin{align*}
x < y &:\Leftrightarrow (x \leq y \land x \neq y)\\
x \geq y &:\Leftrightarrow y \leq x\\
x > y &:\Leftrightarrow y < x
\end{align*}
\end{definition}

\subsection{Propriétés de l'ordre}

\begin{theoreme}[Ordre total]
La relation $\leq$ est un ordre total sur $\R$ : elle est réflexive, antisymétrique, transitive et totale.
\end{theoreme}

\begin{proof}
\textbf{Réflexivité.} Pour $x = [(u_n)]$, on a $u_n \leq u_n$ pour tout $n$ dans $\Q$, donc $x \leq x$.

\textbf{Antisymétrie.} Si $x \leq y$ et $y \leq x$, alors il existe $N_1, N_2$ tels que pour $n \geq N_1$ : $u_n \leq v_n$ et pour $n \geq N_2$ : $v_n \leq u_n$.

Pour $n \geq \max(N_1, N_2)$, on a $u_n = v_n$ dans $\Q$. Donc $(u_n - v_n)$ est nulle à partir d'un certain rang, donc tend vers $0$, donc $(u_n) \sim (v_n)$, donc $x = y$.

\textbf{Transitivité.} Si $x \leq y$ et $y \leq z$, alors il existe $N_1, N_2$ tels que pour $n \geq N_1$ : $u_n \leq v_n$ et pour $n \geq N_2$ : $v_n \leq w_n$.

Pour $n \geq \max(N_1, N_2)$, par transitivité dans $\Q$ : $u_n \leq v_n \leq w_n$, donc $u_n \leq w_n$, donc $x \leq z$.

\textbf{Totalité.} Pour $x = [(u_n)]$ et $y = [(v_n)]$, considérons la suite $(u_n - v_n)$. Cette suite est de Cauchy, donc elle ne peut "osciller indéfiniment".

Plus précisément, soit on a $u_n - v_n \geq 0$ pour une infinité d'indices (auquel cas, par un argument de Cauchy, $u_n - v_n \geq -\varepsilon$ éventuellement, donc $x \geq y - \varepsilon$ pour tout $\varepsilon$, donc $x \geq y$), soit on a $u_n - v_n < 0$ pour une infinité d'indices (auquel cas $y > x$).
\end{proof}

\begin{proposition}[Trichotomie]
Pour tous $x, y \in \R$, exactement une des trois conditions suivantes est vraie :
\[
x < y \quad \text{ou} \quad x = y \quad \text{ou} \quad x > y
\]
\end{proposition}

\subsection{Compatibilité avec les opérations}

\begin{proposition}[Compatibilité avec l'addition]
Pour tous $x, y, z \in \R$ :
\[
x \leq y \Leftrightarrow x + z \leq y + z
\]
\end{proposition}

\begin{proof}
Soient $x = [(u_n)]$, $y = [(v_n)]$, $z = [(w_n)]$.

Si $x \leq y$, il existe $N$ tel que pour $n \geq N$ : $u_n \leq v_n$. Alors pour $n \geq N$ :
\[
u_n + w_n \leq v_n + w_n
\]
dans $\Q$, donc $x + z \leq y + z$.

Réciproquement, si $x + z \leq y + z$, alors $(x + z) + (-z) \leq (y + z) + (-z)$, donc $x \leq y$.
\end{proof}

\begin{proposition}[Compatibilité avec la multiplication]
Pour tous $x, y, z \in \R$ :
\begin{enumerate}
    \item Si $z > 0$ : $x \leq y \Rightarrow xz \leq yz$
    \item Si $z < 0$ : $x \leq y \Rightarrow xz \geq yz$
    \item Si $z = 0$ : $xz = yz = 0$
\end{enumerate}
\end{proposition}

\begin{proof}[Preuve du point (1)]
Soient $x = [(u_n)]$, $y = [(v_n)]$, $z = [(w_n)]$ avec $z > 0$.

Par le lemme \ref{lem:reel_non_nul}, il existe $\varepsilon > 0$ et $N_0$ tels que pour $n \geq N_0$ : $w_n \geq \varepsilon > 0$.

Si $x \leq y$, il existe $N_1$ tel que pour $n \geq N_1$ : $u_n \leq v_n$.

Pour $n \geq \max(N_0, N_1)$, on a $w_n > 0$ et $u_n \leq v_n$, donc $u_n w_n \leq v_n w_n$ dans $\Q$.

Donc $xz \leq yz$.

Les points (2) et (3) se démontrent de manière similaire.
\end{proof}

\section{Complétude de $\R$}

Nous arrivons maintenant à la propriété fondamentale de $\R$, celle qui le distingue de $\Q$ et justifie toute cette construction.

\begin{theoreme}[Complétude de $\R$]\label{thm:completude}
Toute suite de Cauchy dans $\R$ converge dans $\R$.
\end{theoreme}

\begin{proof}[Preuve (esquisse)]
Soit $(x_k)_{k \in \N}$ une suite de Cauchy dans $\R$. Pour chaque $k$, écrivons $x_k = [(u_k^{(n)})_{n \in \N}]$ où $(u_k^{(n)})_{n \in \N}$ est une suite de Cauchy dans $\Q$.

L'idée est de "diagonaliser" : pour chaque $k$, choisissons un élément $u_k \in \Q$ qui représente "bien" $x_k$. Plus précisément, on peut choisir $u_k$ tel que $x_k$ soit proche de la suite constante $\bar{u_k}$.

On peut montrer que la suite $(u_k)_{k \in \N}$ ainsi construite est de Cauchy dans $\Q$. En effet, si $x_k$ et $x_\ell$ sont proches dans $\R$ (car $(x_k)$ est de Cauchy dans $\R$), alors $u_k$ et $u_\ell$ sont proches dans $\Q$.

Posons $L := [(u_k)_{k \in \N}] \in \R$. On peut alors montrer que $x_k \to L$ dans $\R$.

Les détails techniques sont assez délicats et nécessitent de jongler avec plusieurs niveaux d'approximation.
\end{proof}

\begin{corollaire}[Critère de Cauchy]
Une suite dans $\R$ converge si et seulement si elle est de Cauchy.
\end{corollaire}

\begin{proof}
($\Rightarrow$) Si $(x_n)$ converge vers $L$ dans $\R$, alors pour tout $\varepsilon > 0$, il existe $N$ tel que pour $n \geq N$ : $|x_n - L| < \varepsilon/2$. Donc pour $m, n \geq N$ :
\[
|x_m - x_n| \leq |x_m - L| + |L - x_n| < \varepsilon/2 + \varepsilon/2 = \varepsilon
\]

($\Leftarrow$) C'est le théorème \ref{thm:completude}.
\end{proof}

\begin{remarque}
Cette équivalence est \textbf{fausse dans $\Q$} : il existe des suites de Cauchy dans $\Q$ qui ne convergent vers aucun rationnel (exemple : la suite qui "tend vers $\sqrt{2}$"). C'est précisément cette lacune que $\R$ comble.

La complétude de $\R$ est la clé de toute l'analyse mathématique : elle permet de définir rigoureusement les limites, la continuité, la dérivation, l'intégration, les séries, etc.
\end{remarque}

\begin{theoreme}[Propriété de la borne supérieure]
Toute partie non vide et majorée de $\R$ admet une borne supérieure dans $\R$.
\end{theoreme}

\begin{proof}[Idée de la preuve]
Soit $A \subseteq \R$ non vide et majoré. On peut construire une suite de Cauchy $(u_n)$ dont les termes "se rapprochent" de la borne supérieure de $A$ : par dichotomie, on raffine progressivement un encadrement de $\sup A$.

Par complétude de $\R$, cette suite converge vers un élément $s \in \R$. On montre alors que $s = \sup A$.
\end{proof}

\begin{remarque}
Cette propriété est également \textbf{fausse dans $\Q$}. Par exemple, l'ensemble $A = \{x \in \Q \mid x^2 < 2\}$ est non vide et majoré dans $\Q$, mais n'admet pas de borne supérieure dans $\Q$ (car $\sup A$ devrait être $\sqrt{2} \notin \Q$).
\end{remarque}

\section{Densité de $\Q$ dans $\R$}

Bien que $\R$ soit "plus grand" que $\Q$ (au sens où il contient des éléments comme $\sqrt{2}$ qui ne sont pas dans $\Q$), les rationnels sont "denses" dans les réels.

\begin{theoreme}[Densité de $\Q$ dans $\R$]
Pour tous $x, y \in \R$ avec $x < y$, il existe $q \in \Q$ tel que $x < q < y$.
\end{theoreme}

\begin{proof}
Soient $x = [(u_n)]$ et $y = [(v_n)]$ avec $x < y$. Alors il existe $N$ tel que pour $n \geq N$ : $u_n < v_n$ dans $\Q$.

Fixons un tel $n \geq N$. Par densité de $\Q$ dans lui-même, il existe $q \in \Q$ tel que $u_n < q < v_n$.

Montrons que $x < [(q, q, q, \ldots)] < y$.

Pour $[(q, q, q, \ldots)]$ et $x = [(u_n)]$ : pour $k \geq N$, on a $u_k \leq u_n + \varepsilon$ (pour $k$ assez grand, car $(u_n)$ est de Cauchy), et $u_n < q$, donc éventuellement $u_k < q$. Ainsi $x < q$.

De même, $q < y$.
\end{proof}

\begin{corollaire}
Tout nombre réel est limite d'une suite de nombres rationnels.
\end{corollaire}

\begin{proof}
Soit $x \in \R$. Par densité, pour tout $n \in \N^*$, il existe $q_n \in \Q$ tel que :
\[
x - \frac{1}{n} < q_n < x + \frac{1}{n}
\]

Donc $|x - q_n| < \frac{1}{n} \to 0$, donc $q_n \to x$.
\end{proof}

\begin{remarque}
Cette densité montre que $\Q$ est "dense" dans $\R$ : tout réel est approchable arbitrairement bien par des rationnels. Pourtant, $\Q$ n'épuise pas $\R$ : il y a "beaucoup plus" de réels que de rationnels (au sens de la cardinalité : $\Q$ est dénombrable, $\R$ est non dénombrable).
\end{remarque}

\section{Existence de racines}

Maintenant que $\R$ est construit et que nous avons établi sa complétude, nous pouvons démontrer l'existence de racines carrées, répondant ainsi au problème initial.

\begin{theoreme}[Existence de racines carrées]
Pour tout $x \in \R$ avec $x \geq 0$, il existe un unique $y \in \R$ avec $y \geq 0$ tel que $y^2 = x$.
\end{theoreme}

\begin{proof}
\textbf{Existence.} Si $x = 0$, on prend $y = 0$. Supposons $x > 0$.

Construisons une suite de Cauchy qui "converge vers $\sqrt{x}$". On définit la suite $(y_n)$ par :
\[
y_0 = 1, \quad y_{n+1} = \frac{1}{2}\left(y_n + \frac{x}{y_n}\right)
\]

Cette suite est bien définie dans $\Q$ si $x \in \Q$, et dans tous les cas elle satisfait :

1. $y_n > 0$ pour tout $n$ (par récurrence)
2. $y_n \geq \sqrt{x}$ pour $n \geq 1$ : en effet, si $y_n \geq \sqrt{x}$, alors :
\[
y_{n+1} = \frac{1}{2}\left(y_n + \frac{x}{y_n}\right) \geq \frac{1}{2} \cdot 2\sqrt{y_n \cdot \frac{x}{y_n}} = \sqrt{x}
\]
par l'inégalité arithmético-géométrique.

3. $(y_n)$ est décroissante à partir du rang 1 :
\[
y_{n+1} - y_n = \frac{1}{2}\left(\frac{x}{y_n} - y_n\right) = \frac{x - y_n^2}{2y_n} \leq 0
\]
si $y_n \geq \sqrt{x}$.

4. $(y_n)$ est minorée par $\sqrt{x}$ et décroissante, donc elle est de Cauchy (toute suite monotone bornée est de Cauchy).

Par complétude de $\R$, la suite $(y_n)$ converge vers un $y \in \R$.

En passant à la limite dans la relation $y_{n+1} = \frac{1}{2}(y_n + \frac{x}{y_n})$, on obtient :
\[
y = \frac{1}{2}\left(y + \frac{x}{y}\right) \quad \Rightarrow \quad 2y = y + \frac{x}{y} \quad \Rightarrow \quad y = \frac{x}{y} \quad \Rightarrow \quad y^2 = x
\]

\textbf{Unicité.} Si $y, z \geq 0$ satisfont $y^2 = z^2 = x$, alors :
\[
0 = y^2 - z^2 = (y-z)(y+z)
\]

Puisque $y + z > 0$ (car $y, z \geq 0$ et $x > 0$), on a $y - z = 0$, donc $y = z$.
\end{proof}

\begin{notation}
Pour $x \geq 0$, on note $\sqrt{x}$ l'unique réel positif dont le carré vaut $x$.
\end{notation}

\begin{corollaire}
Le nombre $\sqrt{2}$ existe dans $\R$.
\end{corollaire}

\begin{proof}
Appliquer le théorème précédent avec $x = 2$.
\end{proof}

\begin{remarque}
Plus généralement, pour tout $n \in \N^*$ et tout $x \geq 0$, il existe un unique $y \geq 0$ tel que $y^n = x$, noté $\sqrt[n]{x}$ ou $x^{1/n}$. La preuve est similaire, en utilisant la suite :
\[
y_0 = 1, \quad y_{k+1} = \frac{1}{n}\left((n-1)y_k + \frac{x}{y_k^{n-1}}\right)
\]
\end{remarque}

\section{Synthèse et remarques finales}

\subsection{Propriétés fondamentales de $\R$}

\begin{theoreme}[Structure de $\R$]
L'ensemble $\R$ des nombres réels possède les propriétés suivantes :
\begin{enumerate}
    \item L'addition est associative, commutative, possède un élément neutre ($0$), et tout élément possède un opposé
    \item La multiplication est associative, commutative, possède un élément neutre ($1$), et distribue sur l'addition
    \item Tout élément non nul possède un inverse multiplicatif
    \item Il n'existe pas de diviseurs de zéro
    \item La relation $\leq$ est un ordre total compatible avec les opérations
    \item \textbf{$\R$ est complet : toute suite de Cauchy converge}
    \item \textbf{$\Q$ est dense dans $\R$}
    \item \textbf{Toute partie non vide et majorée admet une borne supérieure}
\end{enumerate}
\end{theoreme}

\begin{remarque}
Les propriétés 1 à 5 sont également satisfaites par $\Q$. Les propriétés 6, 7 et 8 sont spécifiques à $\R$ et constituent son essence. En particulier, la propriété 6 (complétude) est celle qui permet de développer toute l'analyse mathématique.
\end{remarque}

\subsection{Caractérisation de $\R$}

\begin{theoreme}[Unicité de $\R$]
$\R$ est l'unique ensemble (à isomorphisme près) qui est :
\begin{itemize}
    \item Un ensemble ordonné où l'ordre est total et compatible avec les opérations
    \item Complet pour la structure uniforme induite par la distance
    \item Contenant $\Q$ comme sous-ensemble dense
\end{itemize}
\end{theoreme}

\begin{remarque}
Ce théorème montre que $\R$ est "inévitable" : quelle que soit la méthode de construction (suites de Cauchy, coupures de Dedekind, etc.), on obtient toujours le même ensemble (à isomorphisme près).
\end{remarque}

\subsection{La hiérarchie des ensembles de nombres}

Nous avons maintenant construit la hiérarchie complète :
\[
\N \subseteq \Z \subseteq \Q \subseteq \R
\]

Chaque extension résout un problème spécifique :
\begin{itemize}
    \item $\Z$ : Rend la soustraction toujours possible (résout $x + a = b$)
    \item $\Q$ : Rend la division par des non-nuls toujours possible (résout $ax = b$ avec $a \neq 0$)
    \item $\R$ : Comble les lacunes, rend toutes les limites de Cauchy existantes
\end{itemize}

\subsection{Propriétés remarquables de $\R$}

\begin{enumerate}
    \item \textbf{Complétude} : Toute suite de Cauchy converge. Ceci permet de définir rigoureusement les limites et la continuité.
    
    \item \textbf{Connexité} : $\R$ ne peut pas être séparé en deux parties ouvertes non vides disjointes. Ceci traduit l'absence de "trous".
    
    \item \textbf{Propriété d'Archimède} : Pour tous $x, y \in \R$ avec $x > 0$, il existe $n \in \N$ tel que $nx > y$.
    
    \item \textbf{Densité de $\Q$} : Entre deux réels distincts, il existe toujours un rationnel.
    
    \item \textbf{Structure topologique} : $\R$ possède une topologie naturelle (la topologie usuelle) qui en fait un espace métrique complet.
\end{enumerate}

\subsection{Cardinalité}

\begin{theoreme}[Cantor, 1891]
$\R$ est non dénombrable : il n'existe pas de bijection entre $\N$ et $\R$.
\end{theoreme}

\begin{remarque}
Ceci contraste avec $\N$, $\Z$ et $\Q$ qui sont tous dénombrables. Il y a donc "beaucoup plus" de nombres réels que de nombres rationnels, bien que $\Q$ soit dense dans $\R$.

Plus précisément :
\begin{itemize}
    \item $|\N| = |\Z| = |\Q| = \aleph_0$ (cardinal dénombrable)
    \item $|\R| = 2^{\aleph_0} = \mathfrak{c}$ (puissance du continu)
\end{itemize}
\end{remarque}

\subsection{Construction alternative : Les coupures de Dedekind}

Il existe une autre construction de $\R$, due à Richard Dedekind (1858), basée sur les "coupures" de $\Q$.

\begin{definition}[Coupure de Dedekind]
Une \textbf{coupure} de $\Q$ est une paire $(A, B)$ où $A, B \subseteq \Q$ satisfont :
\begin{enumerate}
    \item $A \cup B = \Q$ et $A \cap B = \emptyset$
    \item $A$ et $B$ sont non vides
    \item Pour tous $a \in A$ et $b \in B$ : $a < b$
    \item $A$ n'a pas de plus grand élément
\end{enumerate}
\end{definition}

\begin{remarque}
Dans cette approche, un nombre réel est défini comme une coupure. Par exemple :
\begin{itemize}
    \item Le rationnel $q$ correspond à la coupure $(A, B)$ où $A = \{x \in \Q \mid x < q\}$ et $B = \{x \in \Q \mid x \geq q\}$
    \item Le nombre $\sqrt{2}$ correspond à la coupure $(\{x \in \Q \mid x^2 < 2 \lor x < 0\}, \{x \in \Q \mid x^2 \geq 2 \land x > 0\})$
\end{itemize}

Cette construction est conceptuellement différente de celle par suites de Cauchy, mais produit un ensemble isomorphe. La construction par Cauchy est souvent considérée comme plus naturelle pour l'analyse, tandis que celle de Dedekind met davantage en évidence la structure d'ordre.
\end{remarque}

\subsection{Perspectives}

La construction de $\R$ marque l'aboutissement de la hiérarchie des ensembles de nombres pour l'analyse classique. Cependant, on peut aller plus loin :

\begin{itemize}
    \item \textbf{$\C$ (nombres complexes)} : Extension de $\R$ pour résoudre $x^2 + 1 = 0$. Défini par $\C = \{a + bi \mid a, b \in \R\}$ où $i^2 = -1$.
    
    \item \textbf{Quaternions $\mathbb{H}$} : Extension non commutative de $\C$ de dimension 4.
    
    \item \textbf{Octonions $\mathbb{O}$} : Extension non associative de $\mathbb{H}$ de dimension 8.
    
    \item \textbf{Nombres hyperréels} : Extension de $\R$ contenant des infinitésimaux et des infiniment grands, utilisée en analyse non standard.
    
    \item \textbf{Nombres surréels} : Vaste extension de $\R$ contenant tous les nombres ordinaux, utilisée en théorie des jeux combinatoires.
\end{itemize}

\begin{remarque}[Pourquoi s'arrêter à $\R$ pour l'analyse classique ?]
$\R$ possède toutes les propriétés nécessaires et suffisantes pour développer l'analyse mathématique classique :
\begin{itemize}
    \item Définir les limites, la continuité, la dérivation, l'intégration
    \item Étudier les équations différentielles
    \item Développer la théorie de la mesure et l'intégration de Lebesgue
    \item Construire les espaces fonctionnels (espaces de Hilbert, de Banach, etc.)
\end{itemize}

$\R$ représente l'équilibre parfait entre richesse structurelle et maniabilité conceptuelle.
\end{remarque}

\begin{remarque}[Conclusion philosophique]
La construction de $\R$ illustre une démarche fondamentale en mathématiques : face à un problème (les lacunes de $\Q$), on construit un nouvel objet mathématique qui résout ce problème de manière naturelle et élégante. $\R$ n'est pas "découvert" mais **construit** à partir de $\Q$, en utilisant uniquement la théorie des ensembles.

Cette construction montre la puissance de l'abstraction mathématique : des objets aussi concrets que les nombres réels peuvent être définis rigoureusement à partir d'objets plus simples (les rationnels), eux-mêmes définis à partir d'objets encore plus simples (les entiers, les naturels), dans une chaîne logique partant des axiomes de la théorie des ensembles.

Le passage de $\Q$ à $\R$ marque également un saut qualitatif : on passe d'un ensemble dénombrable à un ensemble non dénombrable, d'un ensemble avec des "trous" à un ensemble "continu", d'un ensemble incomplet à un ensemble complet. Ce saut conceptuel est au cœur de toute l'analyse moderne.
\end{remarque}