\chapter{Relation d'équivalence}
\section{Relations d'équivalences}

Soit $X$ un ensemble. Un sous ensemble $R$ de $X^{2}$ est souvent appelé relation binaire sur $X$. Autrement dit, une relation binaire sur $X$ est un ensemble de couples d'éléments de $X$.

\begin{definition}
	Soient $X$ un ensemble et $R$ une relation binaire sur $X$. On dit que $R$ est une relation d'équivalence sur $X$ si les conditions suivantes sont satisfaites:
	\begin{enumerate}
		\item $\forall a \in X,\quad (a,a)\in R$ (on dit que $R$ est réflexive)
		\item $\forall a,b \in X,\quad (a,b)\in R \Rightarrow (b,a)\in R$ (on dit que $R$ est symétrique)
		\item $\forall a,b,c \in X, \quad ((a,b)\in R\land (b,c)\in R)\Rightarrow (a,c)\in R$ (on dit que $R$ est transitive).
	\end{enumerate}
\end{definition}

\begin{definition}
	Soit $R$ une relation d'équivalence sur $X$, soit $a\in X$, on définit la classe d'équivalence de $a$ sous la relation $R$, par \[[a]_R := \{x\in X \mid(x,a)\in R\}\]
\end{definition}

\begin{exemple}
	La relation de divisibilité $\mid$ est une relation d'équivalence
\end{exemple}

\subsection{Résultats élémentaires}

\begin{lemme}
	Soit $R$ une relation d'équivalence sur $X$. Soit $a\in X$, on a que $a\in[a]_R$.
\end{lemme}

\begin{proof}
	Par définition de la classe d'équivalence,
	\[
		[a]_R = \{x \in X \mid (x, a) \in R \}.
	\]
	Comme $R$ est une relation d'équivalence, elle est réflexive, donc $(a, a) \in R$. Ainsi, $a \in [a]_R$.
\end{proof}

\begin{lemme}
	Soit $R$ une relation d'équivalence sur l'ensemble $X$. Soit $a\in X$, on a
	\[[a]_R=\{x\in X\mid (a,x)\in R\}\].
\end{lemme}

\begin{proof}
	Par définition de la relation d'équivalence on a par symétrie,
	\[
		(a, x) \in R \iff (x, a) \in R
	\]
	Donc les deux définitions coïncident.
\end{proof}

\begin{lemme}
	Soit $R$ une relation d'équivalence sur un ensemble $X$. 
	Pour tous $a,b\in X$, on a l'équivalence suivante \[b\in [a]_R \Leftrightarrow a\in [b]_R\].
\end{lemme}

\begin{proof}
	Si $b \in [a]_R$, alors $(b, a) \in R$. Par symétrie de $R$, on a $(a, b) \in R$, donc $a \in [b]_R$.
\end{proof}

\begin{lemme}
	Soit $R$ une relation d'équivalence sur un ensemble $X$. L'union de toutes les classe $[a]_R$ est égale à $X$.
	Cette union est notée $\bigcup\limits_{a \in X} [a]_R$ et définie par l'équivalence ci-dessous
	\[x\in \bigcup\limits_{a \in X} [a]_R \Leftrightarrow \exists a \in X,\:x\in [a]_R \]
\end{lemme}

\begin{proof}
	Par définition, chaque élément $x \in X$ appartient à une classe $[x]_R$, donc $x \in \bigcup_{a \in X} [a]_R$. Ainsi, l'union couvre tout $X$.
\end{proof}

\begin{theoreme} \label{relationdequiv}
	Soit $R$ une relation d'équivalence sur un ensemble $X$. Soient $a,b \in X$, on a que $[a]_R\cap [b]_R=\emptyset$ ou $[a]_R=[b]_R$.
\end{theoreme}

\begin{proof}
	Supposons que $[a]_R \cap [b]_R \neq \emptyset$. Alors il existe $x \in [a]_R \cap [b]_R$, donc $(a, x) \in R$ et $(b, x) \in R$.
	Par transitivité de $R$, on a $(a, b) \in R$, donc $[a]_R = [b]_R$.
\end{proof}

\begin{lemme}
	Soit $R$ une relation d'équivalence sur un ensemble $X$.\\ Pour tous $a,b\in X$, on a
	\begin{enumerate}
		\item $[a]_R=[b]_R \Leftrightarrow [a]_R \cap[b]_R \neq \emptyset$;
		\item $[a]_R=[b]_R \Leftrightarrow (a,b)\in R$.
	\end{enumerate}
\end{lemme}

\begin{proof}
	\begin{enumerate}
		\item Si $[a]_R = [b]_R$, alors leur intersection est évidemment non vide. Réciproquement, si $[a]_R \cap [b]_R \neq \emptyset$, alors il existe $x \in [a]_R \cap [b]_R$. Par le théorème précédent, cela implique $[a]_R = [b]_R$.
		\item Si $[a]_R = [b]_R$, alors $a \in [b]_R$, donc $(a, b) \in R$. Réciproquement, si $(a, b) \in R$, alors $a \in [b]_R$, donc par transitivité $[a]_R = [b]_R$.
	\end{enumerate}
\end{proof}

\subsection{Partition d'un ensemble}

\begin{definition}
	Soit $X$ un ensemble et soit une famille de sous-ensembles\\ $(Y_i)_{i\in I}\subseteq X$. On dit que $(Y_i)_{i\in I}$ forme une partition de $X$ si :
	\begin{enumerate}
		\item $\bigcup_{i\in I}Y_i=X$,
		\item $\forall i,j \in I \quad Y_i\cap Y_j = \emptyset \quad  \lor \quad Y_i=Y_j $.
	\end{enumerate}
\end{definition}

\begin{lemme}
	Soit R une relation d'équivalence sur $X$. La famille des classe d'équivalence de $R$ donnée par $([a]_R)_{a\in X}$ forme une partition de $X$.
\end{lemme}


\begin{proof}
	Nous devons montrer que les classes d'équivalence vérifient les conditions d'une partition :
	\begin{enumerate}
		\item Réunion couvrant $X$ : Chaque $x \in X$ appartient à une classe d'équivalence, donc l'union des classes est $X$.
		\item Disjonction ou égalité : Par le théorème précédent, deux classes sont soit égales, soit disjointes.
	\end{enumerate}
	Ainsi, la famille des classes d'équivalence forme bien une partition de $X$.
\end{proof}
