\chapter{Construction de $\Z$}

\section{Motivation}

L'ensemble $\N$ des entiers naturels possède une structure riche : on peut y additionner et multiplier librement, et ces opérations satisfont les propriétés habituelles d'associativité, de commutativité et de distributivité. Cependant, $\N$ souffre d'une limitation fondamentale : la soustraction n'y est pas toujours définie. L'équation $x + 5 = 3$ n'admet aucune solution dans $\N$, tout comme l'équation générale $x + a = b$ lorsque $a > b$.

Pour remédier à cette situation, nous construisons l'ensemble $\Z$ des \textbf{entiers relatifs}, dans lequel toute équation de la forme $x + a = b$ admettra une solution unique. L'idée fondamentale est d'interpréter chaque entier relatif comme une \textbf{différence formelle} de deux naturels : le nombre $-3$, par exemple, sera représenté par toutes les différences $a - (a+3)$ où $a \in \N$.

\section{Construction par classes d'équivalence}

\subsection{La relation d'équivalence}

\begin{definition}
On définit une relation binaire $\sim$ sur $\N \times \N$ par :
\[
(a,b) \sim (c,d) \quad :\Leftrightarrow \quad a + d = b + c
\]
\end{definition}

\begin{remarque}
L'intuition derrière cette définition est la suivante : si $(a,b)$ représente la différence $a - b$ et $(c,d)$ représente $c - d$, alors ces deux différences sont égales si et seulement si $a - b = c - d$, c'est-à-dire $a + d = b + c$. Nous utilisons cette dernière formulation car elle n'implique que l'addition dans $\N$, évitant ainsi de faire référence à une soustraction non encore définie.
\end{remarque}

\begin{proposition}\label{prop:equiv_Z}
La relation $\sim$ est une relation d'équivalence sur $\N \times \N$.
\end{proposition}

\begin{proof}
Vérifions les trois propriétés caractéristiques.

\textbf{Réflexivité.} Soit $(a,b) \in \N \times \N$. On a $a + b = b + a$ par commutativité de l'addition dans $\N$, donc $(a,b) \sim (a,b)$.

\textbf{Symétrie.} Soient $(a,b), (c,d) \in \N \times \N$ tels que $(a,b) \sim (c,d)$. Alors $a + d = b + c$, d'où $c + b = d + a$ par commutativité, donc $(c,d) \sim (a,b)$.

\textbf{Transitivité.} Soient $(a,b), (c,d), (e,f) \in \N \times \N$ tels que $(a,b) \sim (c,d)$ et $(c,d) \sim (e,f)$. Les hypothèses donnent $a + d = b + c$ et $c + f = d + e$. En additionnant ces égalités membre à membre :
\[
(a + d) + (c + f) = (b + c) + (d + e)
\]
Par associativité et commutativité de l'addition dans $\N$, ceci se réorganise en :
\[
a + c + d + f = b + c + d + e
\]
La loi de simplification dans $\N$ permet d'abord de simplifier par $d$, puis par $c$, obtenant $a + f = b + e$. Ainsi $(a,b) \sim (e,f)$.
\end{proof}

\begin{definition}
Pour $(a,b) \in \N \times \N$, on note $[a,b]$ la classe d'équivalence de $(a,b)$ pour la relation $\sim$ :
\[
[a,b] := \{(c,d) \in \N \times \N \mid (c,d) \sim (a,b)\}
\]
\end{definition}

\begin{definition}[Les entiers relatifs]
On définit l'ensemble des \textbf{entiers relatifs} par :
\[
\Z := (\N \times \N) / \sim \; = \; \{[a,b] \mid (a,b) \in \N \times \N\}
\]
\end{definition}

\begin{remarque}
L'axiome de remplacement de la théorie ZF garantit l'existence de cet ensemble quotient. Intuitivement, chaque classe $[a,b]$ représente l'entier relatif $a - b$. Par exemple :
\begin{itemize}
    \item $[3,0] = [5,2] = [10,7]$ représente l'entier $3$
    \item $[0,3] = [2,5] = [7,10]$ représente l'entier $-3$
    \item $[5,5] = [0,0] = [12,12]$ représente l'entier $0$
\end{itemize}
\end{remarque}

\subsection{Plongement}

\begin{definition}
On définit l'application $\iota : \N \to \Z$ par :
\[
\iota(n) := [n, 0]
\]
\end{definition}

\begin{proposition}
L'application $\iota$ est injective.
\end{proposition}

\begin{proof}
Soient $m, n \in \N$ tels que $\iota(m) = \iota(n)$, c'est-à-dire $[m,0] = [n,0]$. Par définition de l'égalité des classes d'équivalence, cela signifie $(m,0) \sim (n,0)$, donc $m + 0 = 0 + n$, d'où $m = n$.
\end{proof}

\begin{remarque}
Cette injection canonique permet d'identifier $\N$ avec un sous-ensemble de $\Z$. Désormais, par abus de notation, nous écrirons souvent $n$ au lieu de $[n,0]$ pour $n \in \N$, considérant ainsi $\N \subseteq \Z$. Cette identification respecte l'addition et la multiplication, comme nous le verrons.
\end{remarque}

\section{Opérations sur $\Z$}

\subsection{L'opposé}

\begin{definition}
Pour tout $z = [a,b] \in \Z$, on définit l'\textbf{opposé} de $z$ par :
\[
-z := [b,a]
\]
\end{definition}

\begin{lemme}
L'opposé est bien défini : si $[a,b] = [a',b']$, alors $[b,a] = [b',a']$.
\end{lemme}

\begin{proof}
Supposons $[a,b] = [a',b']$, c'est-à-dire $(a,b) \sim (a',b')$. Alors $a + b' = b + a'$, d'où $b + a' = a + b'$ par commutativité. Ceci signifie $(b,a) \sim (b',a')$, donc $[b,a] = [b',a']$.
\end{proof}

\begin{proposition}
Pour tout $z \in \Z$ : $-(-z) = z$.
\end{proposition}

\begin{proof}
Soit $z = [a,b]$. Alors $-z = [b,a]$ et $-(-z) = [a,b] = z$.
\end{proof}

\begin{exemple}
L'opposé échange intuitivement le signe :
\begin{itemize}
    \item $-[3,0] = [0,3]$ représente $-3$
    \item $-[0,5] = [5,0]$ représente $5$
    \item $-[0,0] = [0,0]$ représente $0$
\end{itemize}
\end{exemple}

\subsection{L'addition}

\begin{definition}
On définit l'addition $+ : \Z \times \Z \to \Z$ par :
\[
[a,b] + [c,d] := [a+c, b+d]
\]
\end{definition}

\begin{proposition}
L'addition est bien définie : si $[a,b] = [a',b']$ et $[c,d] = [c',d']$, alors $[a+c, b+d] = [a'+c', b'+d']$.
\end{proposition}

\begin{proof}
Les hypothèses $[a,b] = [a',b']$ et $[c,d] = [c',d']$ signifient $(a,b) \sim (a',b')$ et $(c,d) \sim (c',d')$, c'est-à-dire $a + b' = b + a'$ et $c + d' = d + c'$. En additionnant ces deux égalités :
\[
(a + b') + (c + d') = (b + a') + (d + c')
\]
Par associativité et commutativité dans $\N$, ceci se réécrit :
\[
(a + c) + (b' + d') = (b + d) + (a' + c')
\]
Donc $(a+c, b+d) \sim (a'+c', b'+d')$, c'est-à-dire $[a+c, b+d] = [a'+c', b'+d']$.
\end{proof}

\begin{definition}
On note $0$ ou $0_\Z$ l'élément $[0,0] \in \Z$.
\end{definition}

\begin{theoreme}[Propriétés de l'addition]
L'addition sur $\Z$ satisfait les propriétés suivantes :
\begin{enumerate}
    \item \textbf{Associativité :} $\forall x, y, z \in \Z, \; (x + y) + z = x + (y + z)$
    \item \textbf{Élément neutre :} $\forall x \in \Z, \; 0 + x = x = x + 0$
    \item \textbf{Existence d'un opposé :} $\forall x \in \Z, \; x + (-x) = 0 = (-x) + x$
    \item \textbf{Commutativité :} $\forall x, y \in \Z, \; x + y = y + x$
\end{enumerate}
\end{theoreme}

\begin{proof}
Vérifions chaque propriété.

\textbf{Associativité.} Soient $x = [a,b]$, $y = [c,d]$, $z = [e,f]$. Alors :
\begin{align*}
(x + y) + z &= [a+c, b+d] + [e,f] = [(a+c)+e, (b+d)+f]\\
x + (y + z) &= [a,b] + [c+e, d+f] = [a+(c+e), b+(d+f)]
\end{align*}
Ces expressions sont égales par associativité de l'addition dans $\N$.

\textbf{Neutralité.} Pour $x = [a,b]$ :
\[
0 + x = [0,0] + [a,b] = [0+a, 0+b] = [a,b] = x
\]
La neutralité à droite s'obtient de manière similaire.

\textbf{Opposé.} Pour $x = [a,b]$ :
\[
x + (-x) = [a,b] + [b,a] = [a+b, b+a]
\]
Or $(a+b, b+a) \sim (0,0)$ car $(a+b) + 0 = 0 + (b+a)$ par commutativité dans $\N$. Donc $x + (-x) = [0,0] = 0$.

\textbf{Commutativité.} Pour $x = [a,b]$ et $y = [c,d]$ :
\[
x + y = [a,b] + [c,d] = [a+c, b+d] = [c+a, d+b] = [c,d] + [a,b] = y + x
\]
par commutativité de l'addition dans $\N$.
\end{proof}

\begin{remarque}
Ces quatre propriétés signifient que $(\Z, +)$ possède une structure particulièrement riche : on peut additionner librement dans $\Z$, l'ordre des additions n'importe pas, il existe un élément neutre $0$, et surtout, \textbf{tout élément possède un opposé}. Cette dernière propriété est ce qui distingue fondamentalement $\Z$ de $\N$.
\end{remarque}

\begin{proposition}[Compatibilité du plongement avec l'addition]
Pour tous $m, n \in \N$ : $\iota(m + n) = \iota(m) + \iota(n)$.
\end{proposition}

\begin{proof}
On calcule :
\[
\iota(m) + \iota(n) = [m,0] + [n,0] = [m+n, 0+0] = [m+n, 0] = \iota(m+n)
\]
\end{proof}

\begin{remarque}
Cette proposition affirme que le plongement $\iota : \N \to \Z$ préserve l'addition : additionner dans $\N$ puis plonger dans $\Z$ donne le même résultat que plonger d'abord puis additionner dans $\Z$.
\end{remarque}

\subsection{La multiplication}

\begin{definition}
On définit la multiplication $\cdot : \Z \times \Z \to \Z$ par :
\[
[a,b] \cdot [c,d] := [ac + bd, ad + bc]
\]
\end{definition}

\begin{remarque}
Cette formule peut sembler artificielle, mais elle découle naturellement de l'interprétation intuitive. Si $[a,b]$ représente $a-b$ et $[c,d]$ représente $c-d$, alors leur produit devrait être :
\[
(a-b)(c-d) = ac - ad - bc + bd = (ac+bd) - (ad+bc)
\]
d'où la définition.
\end{remarque}

\begin{proposition}
La multiplication est bien définie.
\end{proposition}

\begin{proof}
Supposons $[a,b] = [a',b']$ et $[c,d] = [c',d']$, c'est-à-dire $a + b' = b + a'$ et $c + d' = d + c'$. Il faut montrer $(ac + bd, ad + bc) \sim (a'c' + b'd', a'd' + b'c')$, c'est-à-dire :
\[
(ac + bd) + (a'd' + b'c') = (ad + bc) + (a'c' + b'd')
\]

Des hypothèses $a + b' = b + a'$ et $c + d' = d + c'$, multiplions la première par $c$ puis par $d$, et la seconde par $a$ puis par $b$ :
\begin{align*}
(a + b')c &= (b + a')c &\Rightarrow& \quad ac + b'c = bc + a'c\\
(a + b')d &= (b + a')d &\Rightarrow& \quad ad + b'd = bd + a'd\\
(c + d')a &= (d + c')a &\Rightarrow& \quad ca + d'a = da + c'a\\
(c + d')b &= (d + c')b &\Rightarrow& \quad cb + d'b = db + c'b
\end{align*}

En additionnant les égalités appropriées et en réorganisant par associativité et commutativité, on obtient le résultat voulu.
\end{proof}

\begin{definition}
On note $1$ ou $1_\Z$ l'élément $[1,0] \in \Z$.
\end{definition}

\begin{theoreme}[Propriétés de la multiplication]
La multiplication sur $\Z$ satisfait les propriétés suivantes :
\begin{enumerate}
    \item \textbf{Associativité :} $\forall x, y, z \in \Z, \; (x \cdot y) \cdot z = x \cdot (y \cdot z)$
    \item \textbf{Commutativité :} $\forall x, y \in \Z, \; x \cdot y = y \cdot x$
    \item \textbf{Élément neutre :} $\forall x \in \Z, \; 1 \cdot x = x = x \cdot 1$
    \item \textbf{Distributivité :} $\forall x, y, z \in \Z, \; x \cdot (y + z) = x \cdot y + x \cdot z$
\end{enumerate}
\end{theoreme}

\begin{proof}
\textbf{Commutativité.} Pour $x = [a,b]$ et $y = [c,d]$ :
\[
x \cdot y = [ac + bd, ad + bc] = [ca + db, cb + da] = y \cdot x
\]
par commutativité de l'addition et de la multiplication dans $\N$.

\textbf{Élément neutre.} Pour $x = [a,b]$ :
\[
1 \cdot x = [1,0] \cdot [a,b] = [1 \cdot a + 0 \cdot b, 1 \cdot b + 0 \cdot a] = [a, b] = x
\]

\textbf{Associativité.} Soient $x = [a,b]$, $y = [c,d]$, $z = [e,f]$. Un calcul direct donne :
\begin{align*}
(x \cdot y) \cdot z &= [ac+bd, ad+bc] \cdot [e,f]\\
&= [(ac+bd)e + (ad+bc)f, (ac+bd)f + (ad+bc)e]\\
&= [ace + bde + adf + bcf, acf + bdf + ade + bce]
\end{align*}
et
\begin{align*}
x \cdot (y \cdot z) &= [a,b] \cdot [ce+df, cf+de]\\
&= [a(ce+df) + b(cf+de), a(cf+de) + b(ce+df)]\\
&= [ace + adf + bcf + bde, acf + ade + bce + bdf]
\end{align*}
Ces expressions sont égales par commutativité de l'addition dans $\N$.

\textbf{Distributivité.} Soient $x = [a,b]$, $y = [c,d]$, $z = [e,f]$. Alors :
\begin{align*}
x \cdot (y + z) &= [a,b] \cdot [c+e, d+f]\\
&= [a(c+e) + b(d+f), a(d+f) + b(c+e)]\\
&= [ac + ae + bd + bf, ad + af + bc + be]
\end{align*}
et
\begin{align*}
x \cdot y + x \cdot z &= [ac+bd, ad+bc] + [ae+bf, af+be]\\
&= [ac+bd+ae+bf, ad+bc+af+be]
\end{align*}
Ces expressions sont égales.
\end{proof}

\begin{proposition}[Compatibilité du plongement avec la multiplication]
Pour tous $m, n \in \N$ : $\iota(m \cdot n) = \iota(m) \cdot \iota(n)$.
\end{proposition}

\begin{proof}
On calcule :
\[
\iota(m) \cdot \iota(n) = [m,0] \cdot [n,0] = [mn + 0 \cdot 0, m \cdot 0 + 0 \cdot n] = [mn, 0] = \iota(mn)
\]
\end{proof}

\begin{proposition}[Multiplication et opposé]
Pour tous $x, y \in \Z$ :
\begin{enumerate}
    \item $(-x) \cdot y = -(x \cdot y) = x \cdot (-y)$
    \item $(-x) \cdot (-y) = x \cdot y$
\end{enumerate}
\end{proposition}

\begin{proof}
Pour $x = [a,b]$ et $y = [c,d]$ :

\textbf{(1)} On a :
\begin{align*}
(-x) \cdot y &= [b,a] \cdot [c,d] = [bc+ad, bd+ac]\\
-(x \cdot y) &= -[ac+bd, ad+bc] = [ad+bc, ac+bd]
\end{align*}
Ces expressions sont égales par commutativité dans $\N$.

\textbf{(2)} On a :
\[
(-x) \cdot (-y) = [b,a] \cdot [d,c] = [bd+ac, bc+ad] = [ac+bd, ad+bc] = x \cdot y
\]
\end{proof}

\begin{remarque}
La seconde propriété exprime la célèbre règle : "moins fois moins donne plus".
\end{remarque}

\begin{theoreme}[Absence de diviseurs de zéro]\label{thm:integrite}
Pour tous $x, y \in \Z$, si $x \cdot y = 0$ alors $x = 0$ ou $y = 0$.
\end{theoreme}

\begin{proof}
Soient $x = [a,b]$ et $y = [c,d]$ tels que $x \cdot y = 0$. Alors :
\[
[ac + bd, ad + bc] = [0,0]
\]
donc $(ac + bd, ad + bc) \sim (0,0)$, c'est-à-dire $ac + bd = ad + bc$.

Supposons $x \neq 0$, montrons que $y = 0$. L'hypothèse $x \neq 0$ signifie $[a,b] \neq [0,0]$, donc $(a,b) \not\sim (0,0)$, c'est-à-dire $a \neq b$. Par trichotomie dans $\N$, soit $a > b$ soit $b > a$.

Si $a > b$, il existe $k \in \N$ non nul tel que $a = b + k$. L'égalité $ac + bd = ad + bc$ devient :
\[
(b+k)c + bd = (b+k)d + bc
\]
En développant : $bc + kc + bd = bd + kd + bc$, d'où $kc = kd$. Par la loi de simplification dans $\N$ (car $k \neq 0$), on obtient $c = d$, donc $y = [c,d] = [c,c] = 0$.

Le cas $b > a$ se traite de manière symétrique.
\end{proof}

\begin{remarque}
Ce théorème affirme qu'il n'existe pas de "diviseurs de zéro" dans $\Z$ : le produit de deux entiers relatifs ne peut être nul que si l'un des deux facteurs est nul. Cette propriété est fondamentale pour l'arithmétique dans $\Z$.
\end{remarque}

\section{Ordre dans $\Z$}

\subsection{Définition de l'ordre}

\begin{definition}
On définit la relation binaire $\leq$ sur $\Z$ par :
\[
\forall x, y \in \Z, \quad x \leq y \; :\Leftrightarrow \; \exists n \in \N, \; x + \iota(n) = y
\]
où $\iota : \N \to \Z$ est le plongement canonique.
\end{definition}

\begin{notation}
Par abus de notation, identifiant $\N$ avec son image par $\iota$ dans $\Z$, on écrit simplement :
\[
x \leq y \; :\Leftrightarrow \; \exists n \in \N, \; x + n = y
\]
Intuitivement, $x \leq y$ signifie que $y$ est obtenu en ajoutant un naturel à $x$.
\end{notation}

\begin{definition}
On définit les relations dérivées :
\begin{align*}
x < y &:\Leftrightarrow (x \leq y \land x \neq y)\\
x \geq y &:\Leftrightarrow y \leq x\\
x > y &:\Leftrightarrow y < x
\end{align*}
\end{definition}

\subsection{Caractérisation via les représentants}

\begin{lemme}\label{lem:order_char}
Soit $z = [a,b] \in \Z$. Alors :
\begin{enumerate}
    \item $z \geq 0 \Leftrightarrow a \geq b$ dans $\N$
    \item $z \leq 0 \Leftrightarrow a \leq b$ dans $\N$
    \item $z > 0 \Leftrightarrow a > b$ dans $\N$
    \item $z < 0 \Leftrightarrow a < b$ dans $\N$
\end{enumerate}
\end{lemme}

\begin{proof}
\textbf{(1)} $z \geq 0$ signifie qu'il existe $n \in \N$ tel que $0 + n = z$, c'est-à-dire $[0,0] + [n,0] = [a,b]$. Donc $[n,0] = [a,b]$, ce qui équivaut à $(n,0) \sim (a,b)$, c'est-à-dire $n + b = 0 + a = a$, donc $n = a - b$. Cette soustraction n'a de sens dans $\N$ que si $a \geq b$, auquel cas $n = a - b$ existe bien.

Réciproquement, si $a \geq b$ dans $\N$, posons $n = a - b \in \N$. Alors $[0,0] + [n,0] = [n,0]$, et $(n,0) \sim (a,b)$ car $n + b = a + 0$, donc $[n,0] = [a,b]$.

\textbf{(2)} $z \leq 0$ signifie qu'il existe $n \in \N$ tel que $z + n = 0$, c'est-à-dire $[a,b] + [n,0] = [0,0]$. Donc $[a+n, b] = [0,0]$, ce qui signifie $(a+n, b) \sim (0,0)$, c'est-à-dire $a + n + 0 = b + 0$, donc $a + n = b$. Ceci implique $a \leq b$ dans $\N$ (avec $n = b - a$).

Réciproquement, si $a \leq b$ dans $\N$, posons $n = b - a \in \N$. Alors $[a,b] + [n,0] = [a+n, b] = [b,b] = [0,0]$ car $(b,b) \sim (0,0)$.

\textbf{(3)} et \textbf{(4)} découlent immédiatement de (1) et (2) en utilisant $z = 0 \Leftrightarrow [a,b] = [0,0] \Leftrightarrow a = b$.
\end{proof}

\begin{lemme}\label{lem:order_alt}
Pour tous $x = [a,b]$ et $y = [c,d]$ dans $\Z$ :
\[
x \leq y \; \Leftrightarrow \; a + d \leq b + c \text{ dans } \N
\]
\end{lemme}

\begin{proof}
$x \leq y$ signifie qu'il existe $n \in \N$ tel que $x + n = y$, c'est-à-dire $[a,b] + [n,0] = [c,d]$. Donc $[a+n, b] = [c,d]$, ce qui signifie $(a+n, b) \sim (c,d)$, c'est-à-dire $(a+n) + d = b + c$, donc $a + d + n = b + c$. Ceci équivaut à l'existence de $n \in \N$ tel que $a + d + n = b + c$, c'est-à-dire $a + d \leq b + c$ dans $\N$.

Réciproquement, si $a + d \leq b + c$ dans $\N$, il existe $n \in \N$ tel que $a + d + n = b + c$. Alors $(a+n) + d = b + c$, donc $(a+n, b) \sim (c,d)$, donc $[a+n, b] = [c,d]$, donc $[a,b] + [n,0] = [c,d]$, donc $x \leq y$.
\end{proof}

\subsection{Propriétés de l'ordre}

\begin{theoreme}[Ordre total]
La relation $\leq$ est un ordre total sur $\Z$ : elle possède les quatre propriétés suivantes :
\begin{enumerate}
    \item \textbf{Réflexivité :} $\forall x \in \Z, \; x \leq x$
    \item \textbf{Antisymétrie :} $\forall x, y \in \Z, \; (x \leq y \land y \leq x) \Rightarrow x = y$
    \item \textbf{Transitivité :} $\forall x, y, z \in \Z, \; (x \leq y \land y \leq z) \Rightarrow x \leq z$
    \item \textbf{Totalité :} $\forall x, y \in \Z, \; x \leq y \lor y \leq x$
\end{enumerate}
\end{theoreme}

\begin{proof}
\textbf{Réflexivité.} Pour tout $x \in \Z$, on a $x + 0 = x$ (avec $0 \in \N$), donc $x \leq x$.

\textbf{Antisymétrie.} Soient $x, y \in \Z$ tels que $x \leq y$ et $y \leq x$. Il existe $m, n \in \N$ tels que $x + m = y$ et $y + n = x$. En substituant la première dans la seconde :
\[
x + m + n = x
\]
Donc $m + n = 0$ dans $\Z$ (par simplification : si $x + a = x + b$ alors $a = b$, ce qui découle des propriétés de l'addition). Or $m + n = \iota(m) + \iota(n) = \iota(m + n)$ (car $\iota$ préserve l'addition). Donc $\iota(m + n) = 0 = \iota(0)$, et par injectivité de $\iota$, on obtient $m + n = 0$ dans $\N$. Ceci n'est possible que si $m = n = 0$, d'où $x = y$.

\textbf{Transitivité.} Soient $x, y, z \in \Z$ tels que $x \leq y$ et $y \leq z$. Il existe $m, n \in \N$ tels que $x + m = y$ et $y + n = z$. Alors :
\[
x + (m + n) = (x + m) + n = y + n = z
\]
Donc $x \leq z$.

\textbf{Totalité.} Soient $x = [a,b]$ et $y = [c,d]$. Par le lemme \ref{lem:order_alt}, $x \leq y \Leftrightarrow a + d \leq b + c$ dans $\N$. Par trichotomie dans $\N$, soit $a + d \leq b + c$, soit $b + c \leq a + d$. Dans le premier cas, $x \leq y$ ; dans le second, $y \leq x$.
\end{proof}

\begin{remarque}
Ces quatre propriétés signifient que $\leq$ est un ordre total sur $\Z$ : on peut comparer n'importe quels deux entiers relatifs, et cette comparaison se comporte de manière cohérente.
\end{remarque}

\subsection{Ordre strict}

\begin{proposition}[Propriétés de l'ordre strict]\label{prop:strict_order}
La relation $<$ possède les trois propriétés suivantes :
\begin{enumerate}
    \item \textbf{Irréflexivité :} $\forall x \in \Z, \; \neg(x < x)$
    \item \textbf{Asymétrie :} $\forall x, y \in \Z, \; x < y \Rightarrow \neg(y < x)$
    \item \textbf{Transitivité :} $\forall x, y, z \in \Z, \; (x < y \land y < z) \Rightarrow x < z$
\end{enumerate}
\end{proposition}

\begin{proof}
\textbf{Irréflexivité.} Par définition, $x < x$ signifierait $x \leq x$ et $x \neq x$, ce qui est impossible.

\textbf{Asymétrie.} Supposons $x < y$ et $y < x$. Alors $x \leq y$ et $y \leq x$, donc $x = y$ par antisymétrie de $\leq$. Mais ceci contredit $x \neq y$ qui découle de $x < y$.

\textbf{Transitivité.} Supposons $x < y$ et $y < z$. Alors $x \leq y$ et $y \leq z$, donc $x \leq z$ par transitivité de $\leq$. De plus, si $x = z$, la transitivité de $\leq$ donnerait $x \leq y \leq x$, donc $x = y$ par antisymétrie, contredisant $x < y$. Ainsi $x \neq z$, d'où $x < z$.
\end{proof}

\begin{proposition}[Caractérisation de l'ordre strict]\label{prop:strict_char}
Pour tous $x, y \in \Z$ :
\[
x < y \; \Leftrightarrow \; \exists n \in \N \setminus \{0\}, \; x + n = y
\]
\end{proposition}

\begin{proof}
($\Rightarrow$) Si $x < y$, alors $x \leq y$ et $x \neq y$. Donc il existe $n \in \N$ tel que $x + n = y$. Si $n = 0$, alors $x = y$, ce qui contredit $x \neq y$. Donc $n \neq 0$.

($\Leftarrow$) Si $x + n = y$ avec $n \neq 0$, alors $x \leq y$. De plus, $x \neq y$ : en effet, si $x = y$, alors $n = 0$ (car dans $\Z$, si $x + a = x$ alors $a = 0$), contradiction. Donc $x < y$.
\end{proof}

\begin{theoreme}[Trichotomie]
Pour tous $x, y \in \Z$, exactement une des trois conditions suivantes est vraie :
\[
x < y \quad \text{ou} \quad x = y \quad \text{ou} \quad x > y
\]
\end{theoreme}

\begin{proof}
Par totalité de $\leq$, on a $x \leq y$ ou $y \leq x$ (ou les deux).

Si $x \leq y$ et $y \leq x$ sont vraies simultanément, alors $x = y$ par antisymétrie.

Si $x \leq y$ est vraie mais $y \leq x$ est fausse, alors nécessairement $x \neq y$ (sinon $y \leq x$ serait vraie par réflexivité), donc $x < y$.

Si $y \leq x$ est vraie mais $x \leq y$ est fausse, alors $y \neq x$, donc $y < x$, c'est-à-dire $x > y$.

L'exclusivité mutuelle découle de l'asymétrie de $<$ : on ne peut avoir simultanément $x < y$ et $y < x$, et si $x = y$ alors ni $x < y$ ni $x > y$ ne peuvent être vraies.
\end{proof}

\begin{lemme}
Pour tous $x = [a,b]$ et $y = [c,d]$ dans $\Z$ :
\[
x < y \; \Leftrightarrow \; a + d < b + c \text{ dans } \N
\]
\end{lemme}

\begin{proof}
Par le lemme \ref{lem:order_alt}, $x \leq y \Leftrightarrow a + d \leq b + c$. De plus, $x = y$ signifie $[a,b] = [c,d]$, donc $(a,b) \sim (c,d)$, donc $a + d = b + c$. Par conséquent :
\[
x < y \Leftrightarrow (x \leq y \land x \neq y) \Leftrightarrow (a + d \leq b + c \land a + d \neq b + c) \Leftrightarrow a + d < b + c
\]
\end{proof}

\subsection{Signe d'un entier}

\begin{definition}
On dit qu'un entier $z \in \Z$ est :
\begin{itemize}
    \item \textbf{positif} (ou \textbf{positif ou nul}) si $z \geq 0$
    \item \textbf{strictement positif} si $z > 0$
    \item \textbf{négatif} (ou \textbf{négatif ou nul}) si $z \leq 0$
    \item \textbf{strictement négatif} si $z < 0$
\end{itemize}
\end{definition}

\begin{proposition}[Trichotomie du signe]
Pour tout $z \in \Z$, exactement une des trois conditions suivantes est vraie :
\[
z > 0 \quad \text{ou} \quad z = 0 \quad \text{ou} \quad z < 0
\]
\end{proposition}

\begin{proof}
Soit $z = [a,b]$. Par trichotomie dans $\N$, soit $a > b$, soit $a = b$, soit $a < b$. Par le lemme \ref{lem:order_char}, ces trois cas correspondent respectivement à $z > 0$, $z = 0$, et $z < 0$. L'exclusivité mutuelle découle de la trichotomie générale.
\end{proof}

\begin{remarque}
Tout entier relatif est donc soit strictement positif, soit nul, soit strictement négatif. Cette partition selon le signe est fondamentale pour l'arithmétique dans $\Z$.
\end{remarque}

\subsection{Compatibilité avec les opérations}

\begin{proposition}[Compatibilité de l'ordre avec l'addition]
Pour tous $x, y, z \in \Z$ :
\[
x \leq y \; \Leftrightarrow \; x + z \leq y + z
\]
et de même pour l'ordre strict :
\[
x < y \; \Leftrightarrow \; x + z < y + z
\]
\end{proposition}

\begin{proof}
Pour le premier point : si $x \leq y$, il existe $n \in \N$ tel que $x + n = y$. Alors :
\[
(x + z) + n = (x + n) + z = y + z
\]
par associativité et commutativité, donc $x + z \leq y + z$.

Réciproquement, si $x + z \leq y + z$, en ajoutant $-z$ à chaque membre :
\[
(x + z) + (-z) \leq (y + z) + (-z)
\]
donc $x \leq y$ par associativité.

Pour le second point : $x < y$ signifie $x \leq y$ et $x \neq y$. Cela équivaut à $x + z \leq y + z$ et $x + z \neq y + z$ (car si $x + z = y + z$ alors $x = y$ en ajoutant $-z$), c'est-à-dire $x + z < y + z$.
\end{proof}

\begin{remarque}
Cette propriété exprime que l'addition "préserve l'ordre" : si on ajoute le même nombre à deux entiers relatifs, leur relation d'ordre ne change pas.
\end{remarque}

\begin{corollaire}
Si $x < y$ et $z < t$, alors $x + z < y + t$.
\end{corollaire}

\begin{proof}
De $x < y$, on obtient $x + z < y + z$. De $z < t$, on obtient $y + z < y + t$. Par transitivité de l'ordre strict, $x + z < y + t$.
\end{proof}

\begin{proposition}[Relation entre ordre et opposé]
Pour tous $x, y \in \Z$ :
\[
x \leq y \; \Leftrightarrow \; -y \leq -x
\]
et de même :
\[
x < y \; \Leftrightarrow \; -y < -x
\]
\end{proposition}

\begin{proof}
Soient $x = [a,b]$ et $y = [c,d]$.

Pour le premier point : $x \leq y$ signifie $a + d \leq b + c$ (lemme \ref{lem:order_alt}). Or $-y = [d,c]$ et $-x = [b,a]$. On a $-y \leq -x$ si et seulement si $d + a \leq c + b$, ce qui est exactement $a + d \leq b + c$ par commutativité dans $\N$.

Le second point découle du premier en utilisant la définition de $<$.
\end{proof}

\begin{remarque}
Cette proposition exprime que prendre l'opposé "renverse l'ordre" : si $x \leq y$ alors $-y \leq -x$.
\end{remarque}

\begin{theoreme}[Compatibilité de l'ordre avec la multiplication]\label{thm:mult_order}
Pour tous $x, y, z \in \Z$ :
\begin{enumerate}
    \item Si $z > 0$ : $x < y \Rightarrow xz < yz$ et $x \leq y \Rightarrow xz \leq yz$
    \item Si $z < 0$ : $x < y \Rightarrow xz > yz$ et $x \leq y \Rightarrow xz \geq yz$
    \item Si $z = 0$ : $xz = yz = 0$
\end{enumerate}
\end{theoreme}

\begin{proof}
\textbf{(1)} Supposons $x < y$ et $z > 0$. Par la proposition \ref{prop:strict_char}, il existe $n \in \N \setminus \{0\}$ tel que $x + n = y$. En multipliant par $z$ :
\[
yz = (x + n)z = xz + nz
\]
par distributivité.

Il faut montrer que $nz > 0$. Écrivons $z = [c,d]$ avec $c > d$ (car $z > 0$ par le lemme \ref{lem:order_char}). Alors :
\[
nz = [n,0] \cdot [c,d] = [nc + 0 \cdot d, nd + 0 \cdot c] = [nc, nd]
\]

Puisque $c > d$ et $n > 0$ dans $\N$, la compatibilité de l'ordre avec la multiplication dans $\N$ donne $nc > nd$. Donc $nz > 0$ par le lemme \ref{lem:order_char}. Par la proposition \ref{prop:strict_char}, $xz < yz$.

Le cas $x \leq y$ se traite de manière analogue.

\textbf{(2)} Si $z < 0$, alors $-z > 0$. Par (1), $x < y \Rightarrow x(-z) < y(-z)$. Or $x(-z) = -(xz)$ et $y(-z) = -(yz)$ d'après les propriétés de la multiplication avec l'opposé. Donc $-(xz) < -(yz)$, ce qui par la proposition précédente donne $yz < xz$, c'est-à-dire $xz > yz$.

\textbf{(3)} Tout élément de $\Z$ multiplié par $0$ donne $0$.
\end{proof}

\begin{remarque}
Ce théorème exprime un fait important : multiplier par un nombre positif préserve l'ordre, mais multiplier par un nombre négatif renverse l'ordre. C'est pourquoi, par exemple, si $x < y$ et qu'on multiplie par $-1$, on obtient $-x > -y$.
\end{remarque}

\begin{corollaire}[Règle des signes pour l'ordre]
Pour tous $x, y \in \Z$ :
\begin{enumerate}
    \item Si $x, y > 0$, alors $xy > 0$
    \item Si $x, y < 0$, alors $xy > 0$
    \item Si $x > 0$ et $y < 0$, alors $xy < 0$
    \item Si $x < 0$ et $y > 0$, alors $xy < 0$
\end{enumerate}
\end{corollaire}

\begin{proof}
\textbf{(1)} On a $0 < x$ et $y > 0$, donc $0 \cdot y < xy$ par le théorème \ref{thm:mult_order}(1). Or $0 \cdot y = 0$, donc $xy > 0$.

\textbf{(2)} Si $x < 0$ et $y < 0$, alors $-x > 0$ et $-y > 0$. Par (1), $(-x)(-y) > 0$. Or $(-x)(-y) = xy$ d'après les propriétés établies précédemment, donc $xy > 0$.

\textbf{(3)} Si $x > 0$ et $y < 0$, alors $-y > 0$. Par (1), $x(-y) > 0$. Or $x(-y) = -(xy)$, donc $-(xy) > 0$, c'est-à-dire $xy < 0$.

\textbf{(4)} Découle de (3) par commutativité de la multiplication.
\end{proof}

\begin{remarque}
Ce corollaire exprime la célèbre \textbf{règle des signes} : le produit de deux nombres de même signe est positif, le produit de deux nombres de signes opposés est négatif.
\end{remarque}

\section{La soustraction}

\begin{definition}
Pour tous $x, y \in \Z$, on définit la \textbf{soustraction} par :
\[
x - y := x + (-y)
\]
\end{definition}

\begin{proposition}
Pour tous $x, y \in \Z$ : $(x - y) + y = x$.
\end{proposition}

\begin{proof}
Par associativité et propriété de l'opposé :
\[
(x - y) + y = (x + (-y)) + y = x + ((-y) + y) = x + 0 = x
\]
\end{proof}

\begin{corollaire}
L'équation $t + y = x$ admet pour unique solution $t = x - y$.
\end{corollaire}

\begin{proof}
Si $t + y = x$, en ajoutant $-y$ des deux côtés : $t + y + (-y) = x + (-y)$, donc $t = x - y$.
Réciproquement, $(x - y) + y = x$ par la proposition précédente.
\end{proof}

\begin{remarque}
Dans $\Z$, contrairement à $\N$, la soustraction est \textbf{toujours définie}. C'est précisément l'objectif que nous nous étions fixé au début de cette construction. Toute équation $x + a = b$ admet maintenant la solution unique $x = b - a = b + (-a)$, quelle que soit la position relative de $a$ et $b$.
\end{remarque}

\section{Relation modulo}

\begin{definition}
	Soit $m$ un entier supérieur ou égale à $2$. On définit la relation d'égalité modulo $n$ dans $\mathbb{Z}$ par $x\equiv _n y$ si et seulement si, il existe $k\in \mathbb{Z}$ tel que $y=x+kn$. On dit alors que $y$ est égale (ou congru) à $x$ modulo $n$ et on note aussi \[x=y \quad mod \; n\]
\end{definition}

\begin{proposition}
	Pour tout entier $n$ supérieur ou égal à $2$, La relation d'égalité modulo $n$ est une relation d'équivalence.
\end{proposition}

\begin{proof}
    Vérifions que la relation $\equiv_n$ satisfait les trois propriétés d'une relation d'équivalence.

\emph{Réflexivité.} Pour tout entier $a$, on peut écrire $a = a + 0 \cdot n$, ce qui montre que $a \equiv_n a$. La relation est donc réflexive.

\emph{Symétrie.} Soient $a, b$ deux entiers tels que $a \equiv_n b$. Par définition, il existe $k \in \mathbb{Z}$ tel que $a = b + kn$. En soustrayant $kn$ de cette égalité, on obtient $b = a + (-k)n$, ce qui établit que $b \equiv_n a$. La relation est donc symétrique.

\emph{Transitivité.} Soient $a, b, c$ trois entiers tels que $a \equiv_n b$ et $b \equiv_n c$. Par définition, il existe deux entiers $k_1$ et $k_2$ tels que $a = b + k_1n$ et $b = c + k_2n$. En substituant l'expression de $b$ dans la première égalité, on obtient
\[
a = c + k_2n + k_1n = c + (k_1 + k_2)n,
\]
ce qui montre que $a \equiv_n c$. La relation est donc transitive.

Ayant vérifié les trois propriétés, on conclut que $\equiv_n$ est une relation d'équivalence sur $\mathbb{Z}$.
\end{proof}

\begin{notation}
	L'ensemble des multiples de $n$ est
	\[
		n\mathbb{Z}=\{nz\mid z\in \mathbb{Z}\}
	\]
\end{notation}

\begin{remarque}
	On peut exprimer la relation $\equiv_n$ à partir de cet ensemble: on a
	\[
		x\equiv_ny \Leftrightarrow y-x \in m\mathbb{Z}\Leftrightarrow m\mid (y-x)
	\]
\end{remarque}

\begin{notation}
	On définit l'ensemble des classes d'équivalence de $\equiv_n$ comme le quotient
	\[
		\mathbb{Z}/\equiv_n=\mathbb{Z}/n\mathbb{Z}
	\]
\end{notation}

\begin{proposition}
	Pour tout entier $n\geq 2$, l'application $f$ qui à $[x]$ pour $x\in \mathbb{Z}$ associe le reste de la division de $x$ par $m$ est en bijection entre $\mathbb{Z}/n\mathbb{Z}$ et \\ \{0,1,...,n-1\} et donc le cardinal de $\mathbb{Z}/n\mathbb{Z}$ est $n$.
\end{proposition}

\begin{proof}
	Il faut d’abord démontrer que l’application est bien définie. Si la division euclidienne de \( x \in \mathbb{Z} \) s’écrit \( x = kn + r \), alors on a \( f([x]) = r \). Si \( x \equiv_n y \), alors il existe \( k' \in \mathbb{Z} \) tel que \( y - x = k'n \). On a donc :
	\[
		y = (k + k')n + r,
	\]
	et donc \( f([y]) = r \). L’image \( r \) est donc indépendante du représentant.

	Montrons maintenant que l’application \( f \) est une bijection. Elle est injective. En effet, si \( f([x]) = f([y]) = r \), pour \( x, y \in \mathbb{Z} \), alors par définition de \( f \), il existe \( k, k' \in \mathbb{Z} \) tels que :
	\[
		x = kn+ r \quad \text{et} \quad y = k'n + r.
	\]
	On a donc :
	\[
		y - x = (k' - k)n,
	\]
	ce qui donne \( y \equiv_n x \), ou encore \( [x] = [y] \).

	L’application \( f \) est surjective : pour tout \( m \in \{0, \dots, n-1\} \), la division euclidienne de \( m \) par \( n\) s’écrit :
	\[
		n = 0n + m.
	\]
	On a donc \( m = f([m]) \).
\end{proof}

\begin{definition}
	L'addition de $\mathbb{Z}/n\mathbb{Z}$ est l'application
	\[
		+:\mathbb{Z}/n\mathbb{Z}\times\mathbb{Z}/n\mathbb{Z} \to\mathbb{Z}/n\mathbb{Z} : ([x],[y])\mapsto[x]+[y]=[x+y].
	\]
\end{definition}

\begin{proposition}
	L'addition dans $\mathbb{Z}/n\mathbb{Z}$ est bien définie. De plus elle satisfait les propriétés suivantes
	$\forall x,y,z\in \mathbb{Z}$
	\begin{enumerate}
		\item $([x]+[y])+[z]=[x]+([y]+[z])$
		\item $[x]+[0]=[0]+[x]=[x]$
		\item $[x]+[-x]=[-x]+[x]=[0]$
		\item $[x]+[y]=[y]+[x]$
	\end{enumerate}
\end{proposition}

\begin{proof}
	Les propriétés sont évidentes au vu de la définition de $+$. Montrons que l'application proposée est indépendante du choix des représentants. Soient $x,y,x',y'\in \mathbb{Z}$ tels que $x\equiv _nx'$ et $y\equiv _ny'$, et montrons que $[x+y]=[x'+y']$. Par définition, il existe $k,k'\in \mathbb{Z}$ tels que $x=x'+kn$ et $y=y'+k'n$. On a alors $x+y=x'+kn+y'+k'n=(x'+y')+(k+k')n$. On a donc $x+y\equiv_n x'+y'$ ou encore $[x+y]=[x'+y']$
\end{proof}

\begin{remarque}
	On verra plus tard que $(\mathbb{Z}/n\mathbb{Z}, +,[0])$ est un groupe commutatif
\end{remarque}

\begin{definition}
	La multiplication de $\mathbb{Z}/n\mathbb{Z}$ est l'application
	\[
		\cdot:\mathbb{Z}/n\mathbb{Z} \times \mathbb{Z}/n\mathbb{Z}\to \mathbb{Z}/n\mathbb{Z}: ([x],[y]) \mapsto [x]\cdot[y]=[x\cdot y]
	\]
\end{definition}

\begin{proposition}
	La multiplication dans $\mathbb{Z}/n\mathbb{Z}$ est bien définie et satisfait :
	\begin{enumerate}
		\item $([x]\cdot[y])\cdot[z]=[x]\cdot([y]\cdot[z])$
		\item $[x]\cdot[1]=[1]\cdot[x]=[x]$
		\item $[x]\cdot[y]=[y]\cdot[x]=[xy]$
	\end{enumerate}
\end{proposition}

\begin{proof}
	De même, les propriétés sont évidentes vu la définition de $\cdot$.Montrons que l'application est bien définie. Si $[x']=[x]$ et $[y']=[y]$, alors il existe $k,k'\in \mathbb{Z}$ tels que $x'=x+km$ et $y'=y+k'n$. On a alors $x'y'=(x+kn)(y+k'n)=xy +(ky+xk'+kk')n$ donc $[xy]=[x'y']$.
\end{proof}

\begin{remarque}
	Nous verrons que $(\mathbb{Z}/n\mathbb{Z},\cdot,[1])$ forme un monoïde commutatif. 
	Et que $(\mathbb{Z} /n \mathbb{Z},+,.)$ forme un anneau commutatif.
\end{remarque}



\section{Synthèse}

\begin{theoreme}[Propriétés fondamentales de $\Z$]
L'ensemble $\Z$ des entiers relatifs possède les propriétés suivantes :
\begin{enumerate}
    \item L'addition est associative, commutative, possède un élément neutre ($0$), et tout élément possède un opposé
    \item La multiplication est associative, commutative, possède un élément neutre ($1$), et distribue sur l'addition
    \item Il n'existe pas de diviseurs de zéro : $xy = 0 \Rightarrow x = 0 \lor y = 0$
    \item La relation $\leq$ est un ordre total
    \item L'ordre est compatible avec l'addition
    \item Multiplier par un nombre positif préserve l'ordre ; multiplier par un nombre négatif renverse l'ordre
    \item Le plongement $\iota : \N \to \Z$ préserve l'addition, la multiplication et l'ordre
\end{enumerate}
\end{theoreme}

\begin{remarque}[Perspective]
La construction de $\Z$ illustre une méthode générale en mathématiques : pour résoudre une équation qui n'admet pas de solution dans une structure donnée, on construit une structure plus large où l'équation devient résoluble, tout en préservant les propriétés essentielles de la structure de départ. 

Cette démarche se poursuivra avec la construction de $\Q$ (pour résoudre $ax = b$ avec $a \neq 0$), puis de $\R$ (pour combler les "lacunes" de $\Q$), et enfin de $\C$ (pour résoudre $x^2 + 1 = 0$). Chaque étape enrichit notre univers numérique tout en conservant les structures antérieures par des plongements successifs qui préservent les opérations et l'ordre.
\end{remarque}

