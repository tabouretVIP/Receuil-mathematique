\chapter{Axiomatisation de Zermolo-Fraenkel}

\section{Le paradoxe de Russell}

Jusqu'au début du XX\,\textsuperscript{e}~siècle, les mathématiciens manipulaient les ensembles de manière dite \emph{naïve}.  
On les concevait d’abord comme des objets géométriques (points, droites, cercles, etc.), puis comme de simples collections d’objets partageant une propriété commune.

C’est le mathématicien allemand \textsc{Georg Cantor} (1845–1918) qui, à partir de 1874, posa les bases de la \emph{théorie naïve des ensembles}.  
Cette approche intuitive permit à Cantor de développer la notion de \emph{cardinalité} et d’étudier différents « types d’infini » — travaux qui révolutionnèrent les fondements des mathématiques modernes.  
Cependant, cette théorie, bien que féconde, se révéla insuffisamment rigoureuse.

\medskip

En 1901, le philosophe et logicien britannique \textsc{Bertrand Russell} (1872–1970) mit en évidence une contradiction au cœur même de la théorie naïve : le \emph{paradoxe de Russell}.  
Considérons l’ensemble suivant :
\[
R := \{\,x \text{ ensemble} \mid x \notin x\,\}.
\]
Autrement dit, $R$ désigne l’ensemble de tous les ensembles qui ne se contiennent pas eux-mêmes.  
La question est alors : $R$ appartient-il à lui-même~?

\medskip

\noindent
Deux cas se présentent :
\begin{itemize}
  \item Si $R \in R$, alors par définition de $R$, on doit avoir $R \notin R$.
  \item Si $R \notin R$, alors, toujours par définition, $R$ satisfait la condition pour appartenir à $R$ ; donc $R \in R$.
\end{itemize}

Dans les deux cas, on aboutit à la contradiction suivante :
\[
R \in R \;\Longleftrightarrow\; R \notin R.
\]
Cette impossibilité logique révèle une faille fondamentale de la théorie naïve : certaines définitions « trop générales » engendrent des paradoxes.

\medskip

Pour surmonter cette difficulté, les mathématiciens entreprirent de fonder la théorie des ensembles sur une base axiomatique rigoureuse.  
En 1908, \textsc{Ernst Zermelo} (1871–1953) proposa une première axiomatisation visant à éviter les paradoxes.  
Cette théorie fut ensuite enrichie en 1922 par \textsc{Abraham Fraenkel} (1891–1965) et, indépendamment, par \textsc{Thoralf Skolem} (1887–1963).  
L’ensemble de ces axiomes constitue la théorie des ensembles \textbf{ZF} (\emph{Zermelo–Fraenkel}).  
Lorsqu’on y ajoute l’axiome du choix, on obtient la théorie \textbf{ZFC}, aujourd’hui la base de la plupart des mathématiques formelles.

\medskip

Cependant, même la théorie ZF ne permet pas de manipuler des « collections trop grandes » — comme l’ensemble de tous les ensembles — car celles-ci ne sont pas des ensembles, mais des \emph{classes propres}.  
Pour traiter de telles collections, d’autres systèmes ont été développés, notamment la théorie \textbf{NBG} (\emph{von Neumann–Bernays–Gödel}), qui étend ZF en introduisant une distinction explicite entre ensembles et classes. Dans ce chapitre, nous allons énoncés les différents axiomes ainsi que leurs conséquences qui nous permettrons d'établir nos premières constructions.

Commençons par donner cinq axiomes dont les énoncés sont relativement aisés à comprendre et qui sont en accords avec l'intuition commune.

\section{Axiomes élémentaires}

\begin{axiome}[d'extensionnalité]
	Deux ensembles possédant les mêmes éléments sont égaux :
	\begin{equation*}
		\forall A,\forall B, \quad [\forall x, \: (x\in A \Leftrightarrow x\in B)] \: \Rightarrow \:A=B
	\end{equation*}
\end{axiome}
Tout naturellement, nous pouvons facilement prouver l'implication réciproque. En effet, si on suppose qu'on a deux ensembles $A$ et $B$ égaux alors par définition de l'égalité en prenant un élément $x$ de $A$ et la formule $\varphi(y)\equiv x\in y$ (qui est bien dans le langage de $ZF$). Nous abtenons que 
\[
\varphi(A)\Leftrightarrow \varphi(B).
\]
C'est à dire que $A=B \Rightarrow \forall x \: (x\in A \Leftrightarrow x\in B)$
\begin{axiome}[de l'ensemble vide]
	Il existe un ensemble qui ne contient aucun éléments :
	\begin{equation*}
		\exists E,\: \forall x,\: x\notin E
	\end{equation*}
\end{axiome}

\begin{proposition}
	Il n'y a qu'un seul ensemble qui ne possède aucun élément.
	On l'appelle ensemble vide et on le note $\varnothing$.
\end{proposition}

\begin{proof}
	Cela résulte des deux axiomes précédant
\end{proof}

\begin{axiome}[de la paire]
	Pour tous ensembles $a$ et $b$, il existe un ensemble, noté $\{a,b\}$
	, admettant comme éléments $a$ et $b$ et rien d'autre.
	\begin{equation*}
		\forall a,\forall b,\quad \exists E,\forall x, [x\in E \Leftrightarrow (x=a \lor x=b)]
	\end{equation*}
\end{axiome}

\begin{axiome}[de la réunion]
	Pour tout ensemble $I$, il existe un ensemble $U$ dont les éléments sont les éléments des éléments $I$.
	\begin{equation*}
		\forall I,\exists U,\forall x, [x\in U \Leftrightarrow \exists i,(i\in I\land x\in i)]
	\end{equation*}
	L'ensemble $U$ est appelé la réunion des éléments de $I$
\end{axiome}
Par exemple, si $I=\{\{a,b\}, \{c,d\},\{d\}\}$ alors $U= \{a,b,c,d\}$. Ce qu'on note $\bigcup I$. De même, pour $I=\{A,B\}$, l'ensemble $U$ est noté $A\cup B$.
\begin{definition}
	On dit qu'un ensemble $A$ est contenu dans un ensemble $B$, ou que $A$ est une partie de $B$ si tout élément de $A$ est élément de $B$
	\begin{equation*}
		\forall x,[x\in A \Rightarrow x\in B]
	\end{equation*}
	On écrit alors $A\subseteq B$
\end{definition}

\begin{remarque}
    De cette définition, il en vient naturellement que deux ensembles $A$ et $B$ sont égaux si et seulement si il sont inclus l'un a l'autre, c'est à dire
    \[
    A=B \Leftrightarrow (A\subseteq B) \land ( B\subseteq A)
    \]
\end{remarque}

\begin{axiome}[de l'ensemble des parties]
	Pour tout ensemble $A$, il existe un ensemble $P$ dont les éléments sont les ensembles contenus dans $A$.
	\begin{equation*}
		\forall A,\exists P, \forall x, [x\in P\Leftrightarrow \forall y, (y\in x \Rightarrow y\in A)]
	\end{equation*}
	Cet ensembles est noté $\partie(A)$.
\end{axiome}

\begin{axiome}[de l'infini]
	Il existe un ensemble $N$ qui vérifie
	\begin{equation*}
		\exists N, (\varnothing \in N\land \forall n,(n\in N\Rightarrow  n\cup \{n\}\in N))
	\end{equation*}
\end{axiome}
Ce qui veut dire qu'il existe un ensemble $N$ qui admet notamment comme éléments $\varnothing,\{\varnothing\},\{\varnothing,\{\varnothing\}\},\dotsc$


\section{Les axiomes techniques}

D'autres axiomes sont nécessaires pour que la théorie puisse être utilisée. Voici donc, en plus des cinq précédents, les axiomes de séparation (aussi appelé axiomes de compréhension),
les axiomes de substitution et l'axiome de fondation. Ces axiomes constituent la théorie ZF, pour Zermolo-Fraenkel. Et lorsque nous ajouterons l'axiome du choix, nous obtiendrons la théorie ZFC.
On considère des expression logique $\mathscr{C}(x,x_1,\dotsc ,x_k)$ en des variables $x,x_1,\dotsc ,x_k$
et en utilisant les symboles $=,\in , \exists, \lor, \land , \neg$, ainsi que ceux qui s'en déduisent comme
$\subseteq, \Rightarrow, \Leftrightarrow,\dotsc $ Pour chaque expression de ce type, on a un axiome de séparation.

\begin{axiome}[de séparation ou de compréhension]
	Pour chaque expression logique $\mathscr{C}(x,x_1,\dotsc ,x_k),$ en des variables $x,x_1,\dotsc ,x_k$, il y a un axiomes de séparation sont voici l'énoncé :
	\begin{equation*}
		\forall x,\forall x_1,\dotsc, x_k,\forall X,\exists Z, \quad [x\in Z \Leftrightarrow (x\in X \land \mathscr{C}(x,x_1,\dotsc , x_k) )]
	\end{equation*}
	L'ensemble $Z$ est appelé l'ensemble des $x\in X$ tels que $\mathscr{C}(x,x_1,\dotsc , x_k)$ et il est noté $\{x\in X \mid \mathscr{C}(x,x_1,\dots, x_k)\}$
\end{axiome}
\begin{definition}
    Soit $A$ un ensemble non vide. Dès lors, il existe $a\in A$. On peut donc définir l'ensemble $I(a,A)$ par $\{x\in a\mid \forall b\in A,x\in b\}$. On va montrer que cet ensemble ne dépend pas du choix de $a$, ce qui nous permettra de l'appeler l'intersection de $A$, et de le noter $\bigcap A$
\end{definition}
\begin{proof}
    Soit $a,a'\in A$. Comme $a'\in A$, il vient $I(a, A)\subseteq I(a',A)$. Par symétrie, $I(a',A)\subseteq I(a,A)$, d'où $I(a,A) = I(a',A)$
\end{proof}

\begin{remarque}
    Pour $C$ un ensemble $\bigcap C$ existe par l'axiome de séparation. En particulier si $C=\{A,B\}$, on note cette ensemble $A\cap B$.
\end{remarque}

\begin{axiome}[de substitution ou de remplacement]
	Pour chaque expression logique de type $\mathscr{C}(x, y,x_1,\dotsc , x_k)$, il y a un axiome de substitution dont voici l'énoncé :
	\begin{align*}
		 & \forall X, \quad [(\forall x\in X), \forall y_1, \forall y_2 (\mathscr{C}(x, y_1,x_1,\dotsc, x_k)\land \mathscr{C}(x, y_2, x_1, \dotsc , x_k)\Rightarrow y_2 = y_1)\Rightarrow \\
		 & \exists Y, \forall y (y\in Y \Leftrightarrow \exists x\in X, \mathscr{C}(x, y,x_1, \dotsc ,x_k))].
	\end{align*}
\end{axiome}

cela implique que,
et c'est probablement ce qu'il faut en retenir,
que si $X$ est un ensemble et $f$ est une expression logique définissant une fonction sur $X$,
alors l'ensemble $\{f(x) \mid x\in X\}$. On obtiendra
le résultat en prenant pour l'expression logique $\mathscr{C}(x,y ,x_1, \dotsc , x_k)$ l'expression $y=f(x)$.

\begin{axiome}[de fondation]
	Tout ensemble non vide $A$ possède un élément $a$ tel que $a\cap A = \varnothing$
\end{axiome}
L’axiome de fondation assure que la relation d’appartenance est bien fondée, excluant toute boucle du type 
$A\in A$ ou chaînes infinies d’appartenance. Il garantit ainsi une hiérarchie claire des ensembles et permet les raisonnements par récurrence sur leur structure.
\section{Conséquences}

\begin{proposition}\label{ens: atoutseul}
	Pour tout ensemble $a$, il existe un ensemble et un seul dont $a$ est le seul élément. Cet ensemble est noté $\{a\}$
\end{proposition}

\begin{proof}
	Il s'agit d'un cas particulier de l'axiome de la paire avec $a=b$.
\end{proof}

\begin{proposition}
	Si on a un nombre fini d'ensemble $x_1, \dotsc , x_k$, il existe un ensemble et un seul sont les éléments sont $x_1, \dotsc, x_k$, et eux seulement. Cet ensemble est noté $\{x_1, \dotsc , x_k\}$
\end{proposition}

\begin{proof}
	Le résultat se construit par itération finie à l’aide des axiomes de la paire et de la réunion.  
	
	En effet, l’axiome de la paire assure que, pour tout couple d’ensembles $x, y$, l’ensemble $\{x, y\}$ existe.  
	En supposant construit un ensemble $Y = \{x_1, \dotsc, x_{k-1}\}$, on applique l’axiome de la paire à $Y$ et $x_k$, ce qui donne $\{Y, x_k\}$.  
	L’axiome de la réunion fournit alors
	\[
		\bigcup \{Y, x_k\} = Y \cup \{x_k\} = \{x_1, \dotsc, x_k\}.
	\]
	L’unicité découle de l’axiome d’extensionnalité.
\end{proof}

\begin{proposition}
	Il n’existe pas d’ensemble $A$ tel que $A \in A$.
\end{proposition}
\begin{proof}
	Si un tel ensemble existait, l’axiome de fondation appliqué à $\{A\}$ fournirait un élément $a\in A$ tel que $a\cap \{A\} = \varnothing$, ce qui contredit $A\in A$.
\end{proof}

\begin{definition}[Complémentaire]
	Soit $E$ un ensemble et $A\subseteq E$.  
	On appelle \emph{complémentaire de $A$ dans $E$} l’ensemble des éléments de $E$ qui n’appartiennent pas à $A$, noté $\complement_E A$ ou $E\setminus A$ :
	\[
	\complement_E A = \{x\in E \mid x\notin A\}.
	\]
\end{definition}

\begin{proposition}[Lois de De Morgan]
	Soient $A,B\subseteq E$. On a les égalités suivantes :
	\[
	\complement_E (A\cup B) = (\complement_E A)\cap(\complement_E B)
	\quad\text{et}\quad
	\complement_E (A\cap B) = (\complement_E A)\cup(\complement_E B).
	\]
\end{proposition}

\begin{proof}
	Soit $x\in E$.  
	\begin{align*}
		x\in \complement_E (A\cup B)
		&\Leftrightarrow x\notin A\cup B \\
		&\Leftrightarrow (x\notin A)\land(x\notin B) \\
		&\Leftrightarrow x\in (\complement_E A)\cap(\complement_E B),
	\end{align*}
	ce qui prouve la première égalité.  
	La seconde s’obtient de façon analogue en remplaçant $\lor$ par $\land$ dans les équivalences logiques.
\end{proof}

\begin{remarque}
	Ces lois traduisent en langage ensembliste les lois de De Morgan de la logique propositionnelle :
	\[
	\neg(P\lor Q)\equiv (\neg P)\land(\neg Q),
	\quad
	\neg(P\land Q)\equiv (\neg P)\lor(\neg Q).
	\]
	Elles illustrent le parallèle étroit entre logique et théorie des ensembles.
    Lorsque le contexte est clair, on note $A^C$ le complémentaire de $A$ dans $E$
\end{remarque}

Dans ce chapitre, nous avons donc énoncé les axiomes qui constituent la théorie ZF et qui nous permettent, de plusieurs manière, de construire des ensembles de manière formelle. Par la suite nous avons établit les opérations ensemblistes basiques qui sont l'intersection, l'union, la complémentarité, l'appartenance et l'inclusion. A partir de cela nous allons maintenant entamer les constructions qui formeront des concepts fondamentaux en mathématique.