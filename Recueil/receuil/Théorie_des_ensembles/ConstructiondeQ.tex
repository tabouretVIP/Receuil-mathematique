\chapter{Construction de $\Q$}

\section{Motivation}

L'ensemble $\Z$ des entiers relatifs a résolu le problème de la soustraction : toute équation de la forme $x + a = b$ y admet une solution unique. Cependant, $\Z$ souffre d'une nouvelle limitation : la division n'y est pas toujours définie. L'équation $2x = 3$ n'admet aucune solution dans $\Z$, tout comme l'équation générale $ax = b$ lorsque $a$ ne divise pas $b$.

Pour remédier à cette situation, nous construisons l'ensemble $\Q$ des \textbf{nombres rationnels}, dans lequel toute équation de la forme $ax = b$ avec $a \neq 0$ admettra une solution unique. L'idée fondamentale est d'interpréter chaque rationnel comme un \textbf{quotient formel} de deux entiers : le nombre $\frac{3}{2}$, par exemple, sera représenté par toutes les fractions $\frac{3k}{2k}$ où $k \in \Z \setminus \{0\}$.

\section{Construction par classes d'équivalence}

\subsection{La relation d'équivalence}

\begin{definition}
On définit une relation binaire $\sim$ sur $\Z \times (\Z \setminus \{0\})$ par :
\[
(a,b) \sim (c,d) \quad :\Leftrightarrow \quad ad = bc
\]
\end{definition}

\begin{remarque}
L'intuition derrière cette définition est la suivante : si $(a,b)$ représente le quotient $\frac{a}{b}$ et $(c,d)$ représente $\frac{c}{d}$, alors ces deux quotients sont égaux si et seulement si $\frac{a}{b} = \frac{c}{d}$, c'est-à-dire $ad = bc$. Nous utilisons cette dernière formulation car elle n'implique que la multiplication dans $\Z$, évitant ainsi de faire référence à une division non encore définie.

Notons que nous excluons $0$ des dénominateurs car la division par zéro n'a pas de sens.
\end{remarque}

\begin{exemple}
Les paires suivantes sont équivalentes :
\begin{itemize}
    \item $(3,2) \sim (6,4)$ car $3 \cdot 4 = 12 = 2 \cdot 6$
    \item $(1,3) \sim (2,6) \sim (-1,-3)$ car $1 \cdot 6 = 6 = 3 \cdot 2$ et $1 \cdot (-3) = -3 = 3 \cdot (-1)$
    \item $(-2,3) \sim (2,-3)$ car $(-2) \cdot (-3) = 6 = 3 \cdot 2$
\end{itemize}
\end{exemple}

\begin{proposition}\label{prop:equiv_Q}
La relation $\sim$ est une relation d'équivalence sur $\Z \times (\Z \setminus \{0\})$.
\end{proposition}

\begin{proof}
Vérifions les trois propriétés caractéristiques.

\textbf{Réflexivité.} Soit $(a,b) \in \Z \times (\Z \setminus \{0\})$. On a $ab = ba$ par commutativité de la multiplication dans $\Z$, donc $(a,b) \sim (a,b)$.

\textbf{Symétrie.} Soient $(a,b), (c,d) \in \Z \times (\Z \setminus \{0\})$ tels que $(a,b) \sim (c,d)$. Alors $ad = bc$, d'où $cb = da$ par commutativité, donc $(c,d) \sim (a,b)$.

\textbf{Transitivité.} Soient $(a,b), (c,d), (e,f) \in \Z \times (\Z \setminus \{0\})$ tels que $(a,b) \sim (c,d)$ et $(c,d) \sim (e,f)$. Les hypothèses donnent $ad = bc$ et $cf = de$. Multiplions la première égalité par $f$ et la seconde par $b$ :
\begin{align*}
adf &= bcf\\
bcf &= bde
\end{align*}
Donc $adf = bde$. Par commutativité, $af \cdot d = be \cdot d$. Puisque $d \neq 0$ et que $\Z$ est intègre (théorème \ref{thm:integrite}), on peut simplifier par $d$ : $af = be$. Ainsi $(a,b) \sim (e,f)$.
\end{proof}

\begin{definition}
Pour $(a,b) \in \Z \times (\Z \setminus \{0\})$, on note $[a,b]$ ou $\frac{a}{b}$ la classe d'équivalence de $(a,b)$ pour la relation $\sim$ :
\[
\left[\frac{a}{b}\right] := \left\{(c,d) \in \Z \times (\Z \setminus \{0\}) \mid (c,d) \sim (a,b)\right\}
\]
\end{definition}

\begin{definition}[Les nombres rationnels]
On définit l'ensemble des \textbf{nombres rationnels} par :
\[
\Q := (\Z \times (\Z \setminus \{0\})) / \sim \; = \; \left\{\frac{a}{b} \mid a \in \Z, b \in \Z \setminus \{0\}\right\}
\]
\end{definition}

\begin{remarque}
L'axiome de remplacement de la théorie ZF garantit l'existence de cet ensemble quotient. Intuitivement, chaque classe $\frac{a}{b}$ représente le nombre rationnel $\frac{a}{b}$. Par exemple :
\begin{itemize}
    \item $\frac{1}{2} = \frac{2}{4} = \frac{3}{6} = \frac{-1}{-2}$ représente le rationnel $\frac{1}{2}$
    \item $\frac{3}{1} = \frac{6}{2} = \frac{-3}{-1}$ représente le rationnel $3$
    \item $\frac{0}{1} = \frac{0}{2} = \frac{0}{-5}$ représente le rationnel $0$
\end{itemize}
\end{remarque}

\subsection{Plongement de $\Z$ dans $\Q$}

\begin{definition}
On définit l'application $\iota : \Z \to \Q$ par :
\[
\iota(n) := \frac{n}{1}
\]
\end{definition}

\begin{proposition}
L'application $\iota$ est injective.
\end{proposition}

\begin{proof}
Soient $m, n \in \Z$ tels que $\iota(m) = \iota(n)$, c'est-à-dire $\frac{m}{1} = \frac{n}{1}$. Par définition de l'égalité des classes d'équivalence, cela signifie $(m,1) \sim (n,1)$, donc $m \cdot 1 = 1 \cdot n$, d'où $m = n$.
\end{proof}

\begin{remarque}
Cette injection canonique permet d'identifier $\Z$ avec un sous-ensemble de $\Q$. Désormais, par abus de notation, nous écrirons souvent $n$ au lieu de $\frac{n}{1}$ pour $n \in \Z$, considérant ainsi $\Z \subseteq \Q$. Cette identification respecte les opérations, comme nous le verrons.
\end{remarque}

\section{Opérations sur $\Q$}

\subsection{L'opposé}

\begin{definition}
Pour tout $q = \frac{a}{b} \in \Q$, on définit l'\textbf{opposé} de $q$ par :
\[
-q := \frac{-a}{b}
\]
\end{definition}

\begin{lemme}
L'opposé est bien défini : si $\frac{a}{b} = \frac{a'}{b'}$, alors $\frac{-a}{b} = \frac{-a'}{b'}$.
\end{lemme}

\begin{proof}
Supposons $\frac{a}{b} = \frac{a'}{b'}$, c'est-à-dire $(a,b) \sim (a',b')$. Alors $ab' = ba'$. En multipliant par $-1$ : $(-a)b' = b(-a')$, ce qui signifie $(-a,b) \sim (-a',b')$, donc $\frac{-a}{b} = \frac{-a'}{b'}$.
\end{proof}

\begin{proposition}
Pour tout $q \in \Q$ : $-(-q) = q$.
\end{proposition}

\begin{proof}
Soit $q = \frac{a}{b}$. Alors $-q = \frac{-a}{b}$ et $-(-q) = \frac{-(-a)}{b} = \frac{a}{b} = q$.
\end{proof}

\begin{remarque}
On peut aussi définir l'opposé par $-\frac{a}{b} = \frac{a}{-b}$, ce qui donne le même résultat :
\[
\frac{-a}{b} = \frac{a}{-b} \quad \text{car} \quad (-a) \cdot (-b) = ab = b \cdot a
\]
\end{remarque}

\subsection{L'addition}

\begin{definition}
On définit l'addition $+ : \Q \times \Q \to \Q$ par :
\[
\frac{a}{b} + \frac{c}{d} := \frac{ad + bc}{bd}
\]
\end{definition}

\begin{remarque}
Notons que $bd \neq 0$ car $b \neq 0$ et $d \neq 0$, et $\Z$ n'a pas de diviseurs de zéro. La fraction $\frac{ad + bc}{bd}$ est donc bien définie.
\end{remarque}

\begin{proposition}
L'addition est bien définie : si $\frac{a}{b} = \frac{a'}{b'}$ et $\frac{c}{d} = \frac{c'}{d'}$, alors $\frac{ad + bc}{bd} = \frac{a'd' + b'c'}{b'd'}$.
\end{proposition}

\begin{proof}
Les hypothèses $\frac{a}{b} = \frac{a'}{b'}$ et $\frac{c}{d} = \frac{c'}{d'}$ signifient $(a,b) \sim (a',b')$ et $(c,d) \sim (c',d')$, c'est-à-dire $ab' = ba'$ et $cd' = dc'$.

Il faut montrer $(ad + bc, bd) \sim (a'd' + b'c', b'd')$, c'est-à-dire :
\[
(ad + bc) \cdot b'd' = bd \cdot (a'd' + b'c')
\]

Développons le membre de gauche :
\[
(ad + bc) \cdot b'd' = adb'd' + bcb'd'
\]

Développons le membre de droite :
\[
bd \cdot (a'd' + b'c') = bda'd' + bdb'c'
\]

Utilisons maintenant les hypothèses. De $ab' = ba'$, en multipliant par $dd'$ :
\[
ab'dd' = ba'dd'
\]

De $cd' = dc'$, en multipliant par $bb'$ :
\[
bb'cd' = bb'dc'
\]

On obtient donc :
\begin{align*}
adb'd' &= ba'dd' = bda'd'\\
bcb'd' &= bb'cd' = bb'dc' = bdb'c'
\end{align*}

En additionnant : $adb'd' + bcb'd' = bda'd' + bdb'c'$, ce qui est exactement ce qu'il fallait démontrer.
\end{proof}

\begin{definition}
On note $0$ ou $0_\Q$ l'élément $\frac{0}{1} \in \Q$.
\end{definition}

\begin{theoreme}[Propriétés de l'addition]
L'addition sur $\Q$ satisfait les propriétés suivantes :
\begin{enumerate}
    \item \textbf{Associativité :} $\forall x, y, z \in \Q, \; (x + y) + z = x + (y + z)$
    \item \textbf{Élément neutre :} $\forall x \in \Q, \; 0 + x = x = x + 0$
    \item \textbf{Existence d'un opposé :} $\forall x \in \Q, \; x + (-x) = 0 = (-x) + x$
    \item \textbf{Commutativité :} $\forall x, y \in \Q, \; x + y = y + x$
\end{enumerate}
\end{theoreme}

\begin{proof}
Vérifions chaque propriété.

\textbf{Associativité.} Soient $x = \frac{a}{b}$, $y = \frac{c}{d}$, $z = \frac{e}{f}$. Alors :
\begin{align*}
(x + y) + z &= \frac{ad + bc}{bd} + \frac{e}{f} = \frac{(ad + bc)f + bde}{bdf}\\
&= \frac{adf + bcf + bde}{bdf}
\end{align*}
et
\begin{align*}
x + (y + z) &= \frac{a}{b} + \frac{cf + de}{df} = \frac{a \cdot df + b(cf + de)}{b \cdot df}\\
&= \frac{adf + bcf + bde}{bdf}
\end{align*}
Ces expressions sont égales.

\textbf{Neutralité.} Pour $x = \frac{a}{b}$ :
\[
0 + x = \frac{0}{1} + \frac{a}{b} = \frac{0 \cdot b + 1 \cdot a}{1 \cdot b} = \frac{a}{b} = x
\]
Par commutativité (à établir), $x + 0 = x$.

\textbf{Opposé.} Pour $x = \frac{a}{b}$ :
\[
x + (-x) = \frac{a}{b} + \frac{-a}{b} = \frac{ab + b(-a)}{b \cdot b} = \frac{ab - ab}{b^2} = \frac{0}{b^2} = 0
\]
car $\frac{0}{b^2} = \frac{0}{1}$ (puisque $0 \cdot 1 = b^2 \cdot 0 = 0$).

\textbf{Commutativité.} Pour $x = \frac{a}{b}$ et $y = \frac{c}{d}$ :
\[
x + y = \frac{ad + bc}{bd} = \frac{cb + da}{db} = \frac{c}{d} + \frac{a}{b} = y + x
\]
par commutativité de l'addition et de la multiplication dans $\Z$.
\end{proof}

\begin{proposition}[Compatibilité du plongement avec l'addition]
Pour tous $m, n \in \Z$ : $\iota(m + n) = \iota(m) + \iota(n)$.
\end{proposition}

\begin{proof}
On calcule :
\[
\iota(m) + \iota(n) = \frac{m}{1} + \frac{n}{1} = \frac{m \cdot 1 + 1 \cdot n}{1 \cdot 1} = \frac{m + n}{1} = \iota(m+n)
\]
\end{proof}

\subsection{La multiplication}

\begin{definition}
On définit la multiplication $\cdot : \Q \times \Q \to \Q$ par :
\[
\frac{a}{b} \cdot \frac{c}{d} := \frac{ac}{bd}
\]
\end{definition}

\begin{remarque}
Cette définition correspond à la règle intuitive de multiplication des fractions : "on multiplie les numérateurs entre eux et les dénominateurs entre eux". Puisque $b \neq 0$ et $d \neq 0$, on a $bd \neq 0$ (car $\Z$ est intègre), donc la fraction est bien définie.
\end{remarque}

\begin{proposition}
La multiplication est bien définie : si $\frac{a}{b} = \frac{a'}{b'}$ et $\frac{c}{d} = \frac{c'}{d'}$, alors $\frac{ac}{bd} = \frac{a'c'}{b'd'}$.
\end{proposition}

\begin{proof}
Les hypothèses signifient $ab' = ba'$ et $cd' = dc'$. En multipliant ces deux égalités membre à membre :
\[
(ab')(cd') = (ba')(dc')
\]
Par commutativité et associativité dans $\Z$ :
\[
(ac)(b'd') = (bd)(a'c')
\]
Donc $(ac, bd) \sim (a'c', b'd')$, c'est-à-dire $\frac{ac}{bd} = \frac{a'c'}{b'd'}$.
\end{proof}

\begin{definition}
On note $1$ ou $1_\Q$ l'élément $\frac{1}{1} \in \Q$.
\end{definition}

\begin{theoreme}[Propriétés de la multiplication]
La multiplication sur $\Q$ satisfait les propriétés suivantes :
\begin{enumerate}
    \item \textbf{Associativité :} $\forall x, y, z \in \Q, \; (x \cdot y) \cdot z = x \cdot (y \cdot z)$
    \item \textbf{Commutativité :} $\forall x, y \in \Q, \; x \cdot y = y \cdot x$
    \item \textbf{Élément neutre :} $\forall x \in \Q, \; 1 \cdot x = x = x \cdot 1$
    \item \textbf{Distributivité :} $\forall x, y, z \in \Q, \; x \cdot (y + z) = x \cdot y + x \cdot z$
\end{enumerate}
\end{theoreme}

\begin{proof}
\textbf{Commutativité.} Pour $x = \frac{a}{b}$ et $y = \frac{c}{d}$ :
\[
x \cdot y = \frac{ac}{bd} = \frac{ca}{db} = y \cdot x
\]
par commutativité dans $\Z$.

\textbf{Élément neutre.} Pour $x = \frac{a}{b}$ :
\[
1 \cdot x = \frac{1}{1} \cdot \frac{a}{b} = \frac{1 \cdot a}{1 \cdot b} = \frac{a}{b} = x
\]

\textbf{Associativité.} Pour $x = \frac{a}{b}$, $y = \frac{c}{d}$, $z = \frac{e}{f}$ :
\[
(x \cdot y) \cdot z = \frac{ac}{bd} \cdot \frac{e}{f} = \frac{(ac)e}{(bd)f} = \frac{a(ce)}{b(df)} = \frac{a}{b} \cdot \frac{ce}{df} = x \cdot (y \cdot z)
\]
par associativité dans $\Z$.

\textbf{Distributivité.} Pour $x = \frac{a}{b}$, $y = \frac{c}{d}$, $z = \frac{e}{f}$ :
\begin{align*}
x \cdot (y + z) &= \frac{a}{b} \cdot \frac{cf + de}{df} = \frac{a(cf + de)}{b \cdot df}\\
&= \frac{acf + ade}{bdf}
\end{align*}
et
\begin{align*}
x \cdot y + x \cdot z &= \frac{ac}{bd} + \frac{ae}{bf} = \frac{ac \cdot bf + bd \cdot ae}{bd \cdot bf}\\
&= \frac{acbf + bdae}{b^2df} = \frac{acf + ade}{bdf}
\]

En simplifiant numérateur et dénominateur, ces expressions sont égales.
\end{proof}

\begin{proposition}[Compatibilité du plongement avec la multiplication]
Pour tous $m, n \in \Z$ : $\iota(m \cdot n) = \iota(m) \cdot \iota(n)$.
\end{proposition}

\begin{proof}
On calcule :
\[
\iota(m) \cdot \iota(n) = \frac{m}{1} \cdot \frac{n}{1} = \frac{mn}{1} = \iota(mn)
\]
\end{proof}

\subsection{L'inverse}

\begin{definition}
Pour tout $q = \frac{a}{b} \in \Q$ avec $a \neq 0$, on définit l'\textbf{inverse} de $q$ par :
\[
q^{-1} := \frac{b}{a}
\]
\end{definition}

\begin{remarque}
L'inverse n'est défini que pour les rationnels non nuls. Si $a = 0$, alors $\frac{b}{a}$ n'est pas défini car on ne peut pas avoir $0$ au dénominateur.
\end{remarque}

\begin{lemme}
L'inverse est bien défini : si $\frac{a}{b} = \frac{a'}{b'}$ avec $a, a' \neq 0$, alors $\frac{b}{a} = \frac{b'}{a'}$.
\end{lemme}

\begin{proof}
Supposons $(a,b) \sim (a',b')$, c'est-à-dire $ab' = ba'$. Alors $ba' = ab'$, ce qui signifie $(b,a) \sim (b',a')$, donc $\frac{b}{a} = \frac{b'}{a'}$.
\end{proof}

\begin{proposition}
Pour tout $q \in \Q \setminus \{0\}$ :
\begin{enumerate}
    \item $q \cdot q^{-1} = 1 = q^{-1} \cdot q$
    \item $(q^{-1})^{-1} = q$
\end{enumerate}
\end{proposition}

\begin{proof}
\textbf{(1)} Soit $q = \frac{a}{b}$ avec $a \neq 0$. Alors :
\[
q \cdot q^{-1} = \frac{a}{b} \cdot \frac{b}{a} = \frac{ab}{ba} = \frac{ab}{ab} = 1
\]
car $(ab, ab) \sim (1, 1)$ (puisque $ab \cdot 1 = 1 \cdot ab$).

\textbf{(2)} Si $q = \frac{a}{b}$ avec $a \neq 0$, alors $q^{-1} = \frac{b}{a}$ et $(q^{-1})^{-1} = \frac{a}{b} = q$.
\end{proof}

\begin{theoreme}[Propriété fondamentale de $\Q$]
Pour tous $x, y \in \Q$, si $x \cdot y = 0$ alors $x = 0$ ou $y = 0$.
\end{theoreme}

\begin{proof}
Soient $x = \frac{a}{b}$ et $y = \frac{c}{d}$ tels que $x \cdot y = 0$. Alors :
\[
\frac{ac}{bd} = \frac{0}{1}
\]
Donc $(ac, bd) \sim (0, 1)$, c'est-à-dire $ac \cdot 1 = bd \cdot 0 = 0$. Ainsi $ac = 0$ dans $\Z$. Par intégrité de $\Z$ (théorème \ref{thm:integrite}), $a = 0$ ou $c = 0$. Si $a = 0$ alors $x = \frac{0}{b} = 0$. Si $c = 0$ alors $y = \frac{0}{d} = 0$.
\end{proof}

\begin{corollaire}
Tout élément non nul de $\Q$ possède un inverse : pour tout $q \in \Q \setminus \{0\}$, il existe un unique $q^{-1} \in \Q$ tel que $q \cdot q^{-1} = 1$.
\end{corollaire}

\begin{remarque}
Cette propriété distingue fondamentalement $\Q$ de $\Z$ : dans $\Q$, on peut diviser par tout élément non nul, alors que dans $\Z$, seuls $1$ et $-1$ ont un inverse.
\end{remarque}

\section{Ordre sur $\Q$}

\subsection{Définition de l'ordre}

\begin{definition}
On définit la relation binaire $\leq$ sur $\Q$ par : pour $x = \frac{a}{b}$ et $y = \frac{c}{d}$ (avec $b, d > 0$),
\[
x \leq y \quad :\Leftrightarrow \quad ad \leq bc \text{ dans } \Z
\]
\end{definition}

\begin{remarque}
Pour que cette définition soit cohérente, il faut s'assurer que les dénominateurs sont positifs. Tout rationnel $\frac{a}{b}$ peut s'écrire avec un dénominateur positif : si $b < 0$, on utilise $\frac{a}{b} = \frac{-a}{-b}$ avec $-b > 0$.
\end{remarque}

\begin{lemme}
L'ordre est bien défini : si $\frac{a}{b} = \frac{a'}{b'}$ (avec $b, b' > 0$) et $\frac{c}{d} = \frac{c'}{d'}$ (avec $d, d' > 0$), alors :
\[
ad \leq bc \; \Leftrightarrow \; a'd' \leq b'c'
\]
\end{lemme}

\begin{proof}
Les hypothèses donnent $ab' = ba'$ et $cd' = dc'$. Multiplions la première par $d$ et la seconde par $b$ :
\begin{align*}
ab'd &= ba'd\\
bcd' &= bdc'
\end{align*}

Supposons $ad \leq bc$. Multiplions par $b'd'$ (qui est strictement positif) :
\[
adb'd' \leq bcb'd'
\]
En utilisant les égalités ci-dessus :
\[
ab'dd' = ba'dd' \quad \text{et} \quad bcb'd' = bdc'b' = bb'dc'
\]
Donc $a'd'(bb') \leq b'c'(bb')$. Puisque $bb' > 0$, on peut simplifier : $a'd' \leq b'c'$.

La réciproque se démontre de manière symétrique.
\end{proof}

\begin{definition}
On définit les relations dérivées :
\begin{align*}
x < y &:\Leftrightarrow (x \leq y \land x \neq y)\\
x \geq y &:\Leftrightarrow y \leq x\\
x > y &:\Leftrightarrow y < x
\end{align*}
\end{definition}

\subsection{Propriétés de l'ordre}

\begin{theoreme}[Ordre total]
La relation $\leq$ est un ordre total sur $\Q$ : elle est réflexive, antisymétrique, transitive et totale.
\end{theoreme}

\begin{proof}
\textbf{Réflexivité.} Pour tout $x = \frac{a}{b}$ (avec $b > 0$), on a $ab = ba$ donc $ab \leq ba$, donc $x \leq x$.

\textbf{Antisymétrie.} Soient $x = \frac{a}{b}$ et $y = \frac{c}{d}$ (avec $b, d > 0$) tels que $x \leq y$ et $y \leq x$. Alors $ad \leq bc$ et $bc \leq ad$ dans $\Z$. Par antisymétrie de $\leq$ dans $\Z$, on a $ad = bc$, donc $(a,b) \sim (c,d)$, donc $x = y$.

\textbf{Transitivité.} Soient $x = \frac{a}{b}$, $y = \frac{c}{d}$, $z = \frac{e}{f}$ (avec $b, d, f > 0$) tels que $x \leq y$ et $y \leq z$. Alors $ad \leq bc$ et $cf \leq de$ dans $\Z$. Multiplions la première inégalité par $f > 0$ et la seconde par $b > 0$ :
\begin{align*}
adf &\leq bcf\\
bcf &\leq bde
\end{align*}
Par transitivité dans $\Z$, $adf \leq bde$. Puisque $d > 0$, on peut simplifier : $af \leq be$, donc $x \leq z$.

\textbf{Totalité.} Soient $x = \frac{a}{b}$ et $y = \frac{c}{d}$ (avec $b, d > 0$). Par trichotomie dans $\Z$, soit $ad \leq bc$, soit $bc \leq ad$. Dans le premier cas $x \leq y$, dans le second $y \leq x$.
\end{proof}

\begin{proposition}[Trichotomie]
Pour tous $x, y \in \Q$, exactement une des trois conditions suivantes est vraie :
\[
x < y \quad \text{ou} \quad x = y \quad \text{ou} \quad x > y
\]
\end{proposition}

\subsection{Compatibilité avec les opérations}

\begin{proposition}[Compatibilité de l'ordre avec l'addition]
Pour tous $x, y, z \in \Q$ :
\[
x \leq y \; \Leftrightarrow \; x + z \leq y + z
\]
\end{proposition}

\begin{proof}
Soient $x = \frac{a}{b}$, $y = \frac{c}{d}$, $z = \frac{e}{f}$ (avec $b, d, f > 0$).

On a $x + z = \frac{af + be}{bf}$ et $y + z = \frac{cf + de}{df}$, avec $bf, df > 0$.

$x \leq y$ signifie $ad \leq bc$. Multiplions par $f > 0$ : $adf \leq bcf$.

$x + z \leq y + z$ signifie $(af + be) \cdot df \leq bf \cdot (cf + de)$, c'est-à-dire :
\[
afdf + bedf \leq bfcf + bfde
\]
En simplifiant : $adf^2 + bdef \leq bcf^2 + bdef$, d'où $adf^2 \leq bcf^2$.

Puisque $f^2 > 0$, ceci équivaut à $ad \leq bc$, donc à $x \leq y$.
\end{proof}

\begin{proposition}[Compatibilité de l'ordre avec la multiplication]
Pour tous $x, y, z \in \Q$ :
\begin{enumerate}
    \item Si $z > 0$ : $x \leq y \Rightarrow xz \leq yz$
    \item Si $z < 0$ : $x \leq y \Rightarrow xz \geq yz$
    \item Si $z = 0$ : $xz = yz = 0$
\end{enumerate}
\end{proposition}

\begin{proof}
\textbf{(1)} Soient $x = \frac{a}{b}$, $y = \frac{c}{d}$, $z = \frac{e}{f}$ (avec $b, d, f > 0$ et $e > 0$).

$x \leq y$ signifie $ad \leq bc$. Multiplions par $e > 0$ : $ade \leq bce$.

$xz = \frac{ae}{bf}$ et $yz = \frac{ce}{df}$. Donc $xz \leq yz$ signifie $ae \cdot df \leq bf \cdot ce$, c'est-à-dire $aedf \leq bcef$. En simplifiant par $f > 0$ : $aed \leq bce$, ce qui est vrai.

\textbf{(2)} et \textbf{(3)} se démontrent de manière similaire.
\end{proof}

\section{La division}

\begin{definition}
Pour tous $x, y \in \Q$ avec $y \neq 0$, on définit la \textbf{division} par :
\[
\frac{x}{y} := x \cdot y^{-1}
\]
\end{definition}

\begin{proposition}
Pour tous $x, y \in \Q$ avec $y \neq 0$ : $\frac{x}{y} \cdot y = x$.
\end{proposition}

\begin{proof}
Par associativité et définition de l'inverse :
\[
\frac{x}{y} \cdot y = (x \cdot y^{-1}) \cdot y = x \cdot (y^{-1} \cdot y) = x \cdot 1 = x
\]
\end{proof}

\begin{corollaire}
L'équation $t \cdot y = x$ avec $y \neq 0$ admet pour unique solution $t = \frac{x}{y}$.
\end{corollaire}

\begin{remarque}
Dans $\Q$, contrairement à $\Z$, la division par tout élément non nul est toujours définie. C'est précisément l'objectif que nous nous étions fixé au début de cette construction. Toute équation $ax = b$ avec $a \neq 0$ admet maintenant la solution unique $x = \frac{b}{a} = b \cdot a^{-1}$.
\end{remarque}

\section{Synthèse}

\begin{theoreme}[Propriétés fondamentales de $\Q$]
L'ensemble $\Q$ des nombres rationnels possède les propriétés suivantes :
\begin{enumerate}
    \item L'addition est associative, commutative, possède un élément neutre ($0$), et tout élément possède un opposé
    \item La multiplication est associative, commutative, possède un élément neutre ($1$), et distribue sur l'addition
    \item Tout élément non nul possède un inverse multiplicatif
    \item Il n'existe pas de diviseurs de zéro
    \item La relation $\leq$ est un ordre total
    \item L'ordre est compatible avec l'addition et la multiplication (avec renversement pour les nombres négatifs)
    \item Le plongement $\iota : \Z \to \Q$ préserve l'addition, la multiplication et l'ordre
\end{enumerate}
\end{theoreme}

\begin{remarque}[Densité de $\Q$]
Une propriété remarquable de $\Q$ est sa \textbf{densité} : entre deux rationnels distincts, il existe toujours un autre rationnel. En effet, si $x < y$ dans $\Q$, alors :
\[
x < \frac{x + y}{2} < y
\]
Cette propriété sera cruciale pour la construction ultérieure de $\R$.
\end{remarque}

\begin{remarque}[Perspective]
Nous avons maintenant construit la hiérarchie $\N \subseteq \Z \subseteq \Q$. Chaque extension résout un problème :
\begin{itemize}
    \item $\Z$ rend la soustraction toujours possible
    \item $\Q$ rend la division (par des non-nuls) toujours possible
\end{itemize}

Cependant, $\Q$ n'est pas encore "complet" : certaines équations comme $x^2 = 2$ n'ont pas de solution dans $\Q$. De plus, $\Q$ présente des "lacunes" (par exemple, il n'y a pas de rationnel dont le carré vaut $2$). La construction de $\R$ comblera ces lacunes et fournira un ensemble où les limites de suites convergentes existent toujours.
\end{remarque}