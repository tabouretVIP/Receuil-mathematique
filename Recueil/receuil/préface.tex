\chapter*{Préface}
\addcontentsline{toc}{chapter}{Préface}

Ce recueil mathématique est conçu comme un voyage à travers les concepts fondamentaux et avancés des mathématiques. Inspiré par le travail remarquable d'Evan Chen dans \emph{An Infinitely Large Napkin}, ce document vise à offrir une vue d'ensemble cohérente et structurée des mathématiques.

\section*{Organisation du document}

Ce recueil est divisé en plusieurs parties thématiques, chacune explorant un domaine spécifique des mathématiques :
\begin{itemize}
    \item \textbf{Partie I} : Théorie des ensembles
    \item \textbf{Partie II} : Algèbre
    \item \textbf{Partie III}: Algèbre linéaire
    \item \textbf{Partie IV} : Topologie
    \item \textbf{Partie V} : Analyse

\end{itemize}

Chaque partie commence par une table des matières locale pour faciliter la navigation.

\section*{Comment utiliser ce document}

Les concepts sont présentés de manière progressive, avec des définitions rigoureuses, des théorèmes importants et de nombreux exemples. Les démonstrations privilégient l'intuition sans sacrifier la rigueur mathématique.

\vspace{1cm}

\begin{flushright}
\itshape
Bonne lecture !\\
Les Auteurs
\end{flushright}

