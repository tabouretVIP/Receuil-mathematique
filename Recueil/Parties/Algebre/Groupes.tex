\chapterimage{chapter_head_1.pdf}
\chapter{Groupes}

\section{Structures algébriques}

La théorie des groupes trouve son origine au début du XIX siècle, principalement
dans les travaux d’Évariste Galois et de Niels Henrik Abel, qui étudiaient les conditions
de résolubilité des équations polynomiales. Galois a introduit la notion de groupe pour
formaliser les symétries des racines d’un polynôme, posant ainsi les bases d’une théorie
devenue centrale en algèbre et dans de nombreuses branches des mathématiques. Dans
le chapitre précédent, la théorie des ensembles ZFC (Zermelo–Fraenkel avec l’axiome
du choix) nous a fourni un cadre logique solide pour définir rigoureusement les objets
mathématiques. La théorie des groupes s’inscrit dans cette continuité : un groupe peut
être vu comme un ensemble muni d’une loi de composition interne satisfaisant certaines
propriétés (associativité, existence d’un élément neutre et d’inverses). L’intérêt de cette
théorie est de permettre l’étude abstraite des symétries et des transformations, qu’elles
soient géométriques, algébriques ou analytiques. Elle constitue ainsi une première étape
vers une compréhension structurelle et unificatrice des mathématiques modernes.

\begin{definition}
    Soit $E$ un ensemble. On appelle loi de composition sur E une application
f de $E \times E$ dans $E$. La valeur $f (x,y)$ de $f$ pour un couple $(x,y) \in E \times E$ s'appelle le
composé de x et de y pour cette loi. Un ensemble muni d'une loi de composition est appelé un
\textbf{magma}.
\end{definition}